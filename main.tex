\documentclass[11pt,article,oneside,a4paper]{memoir}

%% Packages
%% ========

\usepackage{graphicx}

%% many common packages
\input{commonpackages}

%% Some more packages that you may want to use.  Have a look at the
%% file, and consult the package docs for each.
\input{extrapackages}

%% Our layout configuration.
\input{layoutsetup}

%% Theorem environments.  You will have to adapt this for a German
%% thesis.
\input{theoremsetup}

%% Helpful macros.
\input{macrosetup}

%%page layout settings and listing templates etc.
\input{settings}

\title{\textbf{ZNZ HS16 Introduction to Neuroscience I} \\
       Fall 2016\\\normalsize version 1.0}

\author{
	Vanessa Leite
	\vspace{2em}
	\\Repository page: \url{https://github.com/ssinhaleite/znz-intro-to-neuroscience-I-summary}\\
	Contact \href{mailto:vrcleite@gmail.com}{vrcleite@gmail.com} if you have any questions.}
	\thesistype{The Summary of the lectures in 2016}
	\department{ZNZ - Institute of Neuroinformatics, ETH}
	\date{\today}

\begin{document}
\frontmatter


%% DO NOT CHANGE.
\begin{titlingpage}
  \calccentering{\unitlength}
  \begin{adjustwidth*}{\unitlength-24pt}{-\unitlength-24pt}
    \maketitle
  \end{adjustwidth*}
\end{titlingpage}

\mainmatter

%% This change is needed if the article option for the memoir document class
%% is used, in order to count sections (article) as if they were chapters (memoir)
\counterwithout{section}{chapter}

%% Our content

\newpage
\clearpage
\pagenumbering{roman}
\setcounter{tocdepth}{3}
\setcounter{secnumdepth}{2}
\tableofcontents

\clearpage
\pagenumbering{arabic}

\newpage

\section{Human \& Comparative Neuroanatomy}
\subsection{Human Neuroanatomy}
\subsubsection{Nervous system}
The nervous system is divided in two parts: the Central Nervous System (CNS) and
the Peripheral Nervous System (PNS).
\begin{itemize}
\item CNS
\subitem Brain
\subitem Spinal Cord
\item PNS
\subitem Somatic and autonomic nervous system
\end{itemize}

Both system contains gray and white matter.
In the PNS the gray matter contains \textbf{ganglia}: collection of neuron cell
bodies -, the white matter contains \textbf{nerves}: bundles of axons.
In the CNS the gray matter is divided in:
\begin{itemize}
\item Neural cortex - gray matter on the surface of the brain
\item Nuclei - collection of neuron cell bodies in the interior of CNS
\item Centers - collection of neuron cell bodies in CNS, each center has specific processing functions
\item High centers - the most complex centers in brain.
\end{itemize}
The white matter in CNS is divided in two parts: the {tracts}: bundle of CNS axons
that share a common origin and destination -, and the {columns}: several tracts
that form an anatomically distinct mass

The centers and tracts that connect the brain with other organs and system in
the body are called \textbf{pathways}. The ascending (sensory) pathway is called afferent.
The descending (motor) pathway is called efferent.

\begin{figure}
  \includegraphics[width=\linewidth]{imgs/viewsOfTheBrain.png}
  \caption{Views of the Brain}
  \label{fig:brain}
\end{figure}

Figure \ref{fig:brain} shows some views of the brain.

\subsection{Comparative Neuroanatomy}
\section{Molecular \& Cellular Neuroscience}
\subsection{Building a central nervous system}
\subsection{Excitability}
\subsection{Glia and more}
\subsection{Synapses}

\section{Systems Neuroscience}
\subsection{Somatosensory and Motor Systems}
\subsection{Visual System}
\subsection{Auditory \& Vestibular System}
\subsection{Circuits underlying Emotion}
\subsection{Learning in artificial and biological neural networks}

\newpage

\subsection{References}
The pictures used in this summary are from the following books and slide sets
and belong to their respective owners. In the context of the summary they are
used for educational purposes only.
\begin{itemize}
\item
\item
\end{itemize}
\end{document}
