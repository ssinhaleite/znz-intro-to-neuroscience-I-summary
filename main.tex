\documentclass[12pt,article,oneside,a4paper]{memoir}

%% Packages
%% ========
\usepackage{graphicx}
\usepackage{titlesec}
\usepackage{wrapfig}

\setcounter{secnumdepth}{4}

\titleformat{\paragraph}
{\normalfont\normalsize\bfseries}{\theparagraph}{1em}{}
\titlespacing*{\paragraph}
{0pt}{3.25ex plus 1ex minus .2ex}{1.5ex plus .2ex}

%% many common packages
\input{commonpackages}

%% Some more packages that you may want to use.  Have a look at the
%% file, and consult the package docs for each.
\input{extrapackages}

%% Our layout configuration.
\input{layoutsetup}

%% Theorem environments.  You will have to adapt this for a German
%% thesis.
\input{theoremsetup}

%% Helpful macros.
\input{macrosetup}

%%page layout settings and listing templates etc.
\input{settings}

\title{\textbf{ZNZ HS16 Introduction to Neuroscience I} \\
       Fall 2016\\\normalsize version 1.0}

\author{
	Vanessa Leite
	\vspace{2em}
	\\Repository page: \url{https://github.com/ssinhaleite/znz-intro-to-neuroscience-I-summary}\\
	Contact \href{mailto:vrcleite@gmail.com}{vrcleite@gmail.com} if you have any questions.}
	\thesistype{The Summary of the lectures in 2016}
	\department{ZNZ - Institute of Neuroinformatics, ETH}
	\date{\today}

\begin{document}
\frontmatter


%% DO NOT CHANGE.
\begin{titlingpage}
  \calccentering{\unitlength}
  \begin{adjustwidth*}{\unitlength-24pt}{-\unitlength-24pt}
    \maketitle
  \end{adjustwidth*}
\end{titlingpage}

\mainmatter

%% This change is needed if the article option for the memoir document class
%% is used, in order to count sections (article) as if they were chapters (memoir)
\counterwithout{section}{chapter}

%% Our content

\newpage
\clearpage
\pagenumbering{roman}
\setcounter{tocdepth}{3}
\setcounter{secnumdepth}{2}
\tableofcontents

\clearpage
\pagenumbering{arabic}

\newpage

\section{Human \& Comparative Neuroanatomy}
\subsection{Human Neuroanatomy}
\subsubsection{Why do we need to know the brain}
The famous case of the HM pacient: Henry Gustav Molaison went through a surgery on brain to cure his epilepsy. However, during the surgery two holes were drilled in the front of his skull and a portion of his brain, the front half of the hippocampus on both sides, and most of the almond-shaped amygdala, was sucked out. The procedure went badly wrong and Henry, then aged 27, was left with no ability to store or retrieve new experiences. He lived the subsequent 55 years of his life, until his death in 2008, in the permanent present moment.

\subsubsection{Nervous system}
The nervous system is divided in two parts: the Central Nervous System (CNS) and
the Peripheral Nervous System (PNS). Each part has its own divisions as we can see in Figure \ref{fig:nervousSystem}.

\begin{figure}
  \includegraphics[width=\linewidth]{imgs/division_nervous_system.jpg}
  \caption{Division of the Nervous System}
  \label{fig:nervousSystem}
\end{figure}

\begin{itemize}
\item CNS
\subitem Brain
\subitem Spinal Cord
\item PNS
\subitem Somatic and autonomic nervous system
\end{itemize}

Both system contains gray and white matter.
In the PNS the gray matter contains \textbf{ganglia}: collection of neuron cell
bodies -, the white matter contains \textbf{nerves}: bundles of axons.
In the CNS the gray matter is divided in:
\begin{itemize}
\item Neural cortex - gray matter on the surface of the brain
\item Nuclei - collection of neuron cell bodies in the interior of CNS
\item Centers - collection of neuron cell bodies in CNS, each center has specific processing functions
\item High centers - the most complex centers in brain.
\end{itemize}
The white matter in CNS is divided in two parts: the \textbf{tracts or fasciculus}: bundle of CNS axons
that share a common origin and destination -, and the \textbf{columns or funiculus}: several tracts (fasciculi) that form an anatomically distinct mass

The centers and tracts that connect the brain with other organs and system in
the body are called \textbf{pathways}. The ascending (sensory) pathway is called afferent.
The descending (motor) pathway is called efferent.

Figure \ref{fig:brain} shows the macro division of the brain: Telencephalon, Diencephalon, Brain stem (Midbrain or Mesencephalon, Pons and Cerebellum and Medulla oblongata) and Medulla spinallis), in \ref{fig:brain2} we can see some views of the brain.
Also part of the anatomy of the brain: cranial nerves, meninges, ventricles / cerebrospinal fluid and cerebral circulation.

\begin{figure}
  \includegraphics[width=\linewidth]{imgs/viewsOfTheBrain.png}
  \caption{Division of the brain}
  \label{fig:brain}
\end{figure}

\begin{figure}
  \includegraphics[width=\linewidth]{imgs/viewsOfTheBrain2.png}
  \caption{Views of the Brain}
  \label{fig:brain2}
\end{figure}

\paragraph{Telencephalon - or Forebrain}
The telencephalon (the biggest part of the brain) is divided in lobes, functional cortical areas, basal ganglia and limbic system.

The four lobes (frontal, occitopital, temporal and parietal) are presented in Figure \ref{fig:boundaries}.

\subparagraph{Gray matter}
The macroscopic boundaries of the gray matter are Gyri, Sulci and Commissural fiber tracts. Each one is divided as follows:
\begin{itemize}
\item Gyri
\subitem precentral gyrus
\subitem postcentral gyrus
\subitem pars triangularis
\subitem angular gyrus
\subitem cingulate gyrus
\subitem parahippocampal gyrus
\item Sulci
\subitem central sulcus
\subitem lateral fissure
\subitem parieto-occipital sulcus
\subitem calcarine sulcus
\item Commissural fiber tracts
\subitem corpus callosum
\subsubitem Rostrum
\subsubitem Genu
\subsubitem Truncus
\subsubitem Splenium
\subitem anterior commissure
\end{itemize}

Figure \ref{fig:boundaries} shows the macroscopic boundaries of the gray matter. Besides the anatomical division, there is a functional division of the brain, where each area in the cerebral cortex has specific functional activities. The Wernicke's (language comprehension) and Broca's (speech production) areas are highlited in Figure \ref{fig:functionalBoundaries}.

In 1909 Korbinian Brodmann described areas of the cerebral cortex on the basis of
cytoarchitectural criteria. Areas differ in celltypes, layering and cell distribution, resulting in 52 Brodman Areas.

The human brain is gyrencephalic, i.e, is formed by giri, as the elephant brain. However other species can be  lissencephalic (the brain is smooth, without giri) as the domestic rabbit and the house mouse. Defects in the neuronal migration during early to mid gestation (12th to 24th weeks) leading to impaired development of gyri and sulci.

\begin{figure}
  \includegraphics[width=\linewidth]{imgs/macroscopic_boundaries.png}
  \caption{Macroscopic Boundaries - gray matter of the cortex}
  \label{fig:boundaries}
\end{figure}

\begin{figure}
  \centering
  \includegraphics[width=12cm]{imgs/functional_boundaries.png}
  \caption{Functional Division - gray matter of the cortex}
  \label{fig:functionalBoundaries}
\end{figure}

\newpage

\subparagraph{White matter}

\begin{wrapfigure}[11]{r}{0.5\textwidth}
	\centering
  	\includegraphics[width=5cm]{imgs/macroscopic_whiteMatter.png}
	\caption{White matter - macroscopic fibers}
  	\label{fig:macroscopic_whiteMatter}
\end{wrapfigure}

The white matter can be divided macroscopically and microscopically. Macroscopically we talk about fibers and microscopically we talk about cells. Figure \ref{fig:macroscopic_whiteMatter} exemplify the macroscopic division and Figure \ref{fig:microscopic_whiteMatter} exemplify the microscopic division, where we can see microglias, astrocyte and oligodendrocytes cells.

\begin{itemize}
\item Comissural fibers (red): link areas between the two hemispheres (corpus callosum, anterior commissure, posterior commissure)
\item Association fibers (green): link cortical areas of the same hemisphere.
\item Projecting fibers (blue): link the cortex with subcortical areas of the brain and the spinal cord.
\end{itemize}

\begin{figure}
  \centering
  \includegraphics[width=12cm]{imgs/microscopic_whiteMatter.png}
  \caption{White Matter - microscopic structures}
  \label{fig:microscopic_whiteMatter}
\end{figure}

\subparagraph{Basal Ganglia}

The basal ganglia are the principal subcortical components of a family of neuronal circuits which link the thalamus and cerebral cortex. It is crucial for the initiation and modulation of voluntary movement by sending their output to the motor cortex via the thalamus. In addition, the basal ganglia also contribute to a variety of behavioral and cognitive functions other than voluntary movement.

The basal ganglia is divided in:
\begin{itemize}
\item Striatum: is the major recipient of inputs from the substantia nigra, cerebral cortex, thalamus, and brain stem. In humans (and most primates) consist of the caudate nucleus, the putamen, and the nucleus accumbens. In rats and mices consist of caudate putamen (human caudate nucleus + putamen) and nucleus accumbens.
\item Globus Palidus: is divided into external and internal segments.
\subitem The internal segment (GPi) sends projections to the thalamus and pedunculopontine nucleus (a group of cells located in the brain stem).
\subitem The external segment (GPe) sends projections to the internal segment of the globus pallidus and to the subthalamic nucleus.
\item Susbtantia Nigra: is a midbrain (mesencephalon) structure and contains a dense population of dopamine cells. The substantia nigra can be subdivided into substantia nigra pars compacta and pars reticulata.
\end{itemize}

One of the disorders of the basal ganglia is the parkinson's disease, where the
dopaminergic cells in the substantia nigra pars compacta are lost, it impairs motor skills, speech and other functions.

\subparagraph{Limbic system}
Divided in cingulate gyrus (superior portion of limbic lobe), parahippocampal gyrus (inferior portion of limbic lobe), hippocampus and amigdalar complex.
In Alzheimer's disease, the hippocampus is one of the first regions of the brain to suffer damage; memory problems (especially spatial memories) and disorientation appear among the first symptoms. People with extensive, bilateral hippocampal damage (such as in patients with progressed AD) may experience anterograde amnesia (the inability to form or retain new memories).
The amigdala is envolved in emotions.

\paragraph{Diencephalon}
The diencephalon is divided in thalamus, hipothalamus, epithalamus, subthalamus

\subparagraph{Thalamus}
The thalamus is the gatekeeper of the brain: it is important for the transfer of information from the periphery to sensory processing regions in the telencephalon. It has important gating (filtering) functions: it determines whether sensory information reaches conscious awareness in the neocortex and participates in the integration of motor information from the cerebellum and basal ganglia and transmits this information to cerebral areas concerned with movement.

\subparagraph{Hipothalamus}
The hipothalamus regulates several behaviors that are essential for homeostasis
and reproduction: growth, eating, drinking and maternal behavior, by regulating hormonal secretions from the pituitary gland. It is an important control center for the autonomic nervous system and for the hypothalamus-pituitary-adrenal (HPA) stress-response system.

\subparagraph{Neuroendocrinollogy of Hipothalamus}
\begin{enumerate}
\item Hipothalamus produces releasing hormones (rh) and inhibiting hormones (ih) that directly influence anterior pituitary hormone secretion.
\item Hipothalamus produces two hormones (oxytocin and antidiuretic hormone) that are stored in the posterior pituitary.
\item Hupothalamus overseesthe ANS (?)thereby helping to stimulate the adrenal medulla via sympathetic innervation.
\end{enumerate}

\subparagraph{Epithalamus}
epithelial roof of the third ventricle, habenula, pineal body and afferent/efferent connections. It is responsible for the secretion of melatonin, regulation of day-night cycles, information processing related to olfaction.

\subparagraph{Subthalamus}
It is the continuation of the tegmentum. Functionally part of the basal ganglia (motor control).

\newpage

\paragraph{Mesencephalon - or Midbrain}
\begin{wrapfigure}[11]{r}{0.5\textwidth}
	\centering
  	\includegraphics[width=7cm]{imgs/mesencephalon.png}
	\caption{Mesencephalon - Functional units}
  	\label{fig:mesencephalon}
\end{wrapfigure}

The midbrain is a portion of the CNS associated with vision, hearing, motor control, sleep/wake, arousal (alertness), and temperature regulation. It comprises the tectum (or corpora quadrigemina), tegmentum, the cerebral aqueduct (or ventricular mesocoelia or "iter"), and the cerebral peduncles, as well as several nuclei and fasciculi. Caudally the midbrain adjoins the metencephalon (afterbrain) (pons and cerebellum); while rostrally it adjoins the diencephalon (thalamus, hypothalamus, etc). In Figure  \ref{fig:mesencephalon} the parts of the midbrain are listed.

\begin{enumerate}
\item Tectum (roof)
\subitem superior colliculus: visual and occulomotor reflexes
\subitem inferior colliculus: relay auditory tract
\item Tegmentum (floor)
\item Reticular formation: automatic processing of incoming sensation and outgoing motor
commands, helps to maintain consciousness, can initiate motor response to stimuli (see also
medulla oblongata!)
\item Red nucleus: involuntary control of background muscle tone and limb posture
\item Substantia nigra: regulates activity in the basal nuclei, degeneration of dopaminergic cells causes Parkinson’s disease
\item Cerebral peduncles: connect primary motor cortex with motor neurons in brain and spinal cord, carry ascending sensory information to thalamus
\item Ventral tegmental area (VTA): part of the limbic system, projects e.g. to nucleus accumbens and amygdala, emotional reinforcement.
\end{enumerate}

\paragraph{Pons}
\begin{wrapfigure}[7]{r}{0.5\textwidth}
	\centering
  	\includegraphics[width=7cm]{imgs/pons.png}
	\caption{Pons}
  	\label{fig:mesencephalon}
\end{wrapfigure}

Divded in two parts: locus coeruleus and pontine nuclei. The locus coeruleus (or blue spot) contains noradrenergic cells innervating large portions of the brain, mediating physiological response to panic and stress. The pontine nuclei receive fibers from all cortical areas and relay to the contralateral cerebellum.

\paragraph{Medulla oblongata}
\begin{figure}
	\centering
  	\includegraphics[width=\linewidth]{imgs/medullaOblongata.png}
	\caption{Medulla Oblongata}
  	\label{fig:medullaOblongata}
\end{figure}

It contains four main parts: olives, pyramid, reticular formation and reflex centers.\subparagraph{olive} relay nucleus for afferent connection from motor cortex and red nucleus, efferent to contralateral cerebellum.
\subparagraph{pyramid} contains descending cortico-spinal fibers.
\subparagraph{reticular formation (entire brain stem!)} containing the raphe nuclei and magno/parvocellular nuclei, which regulate respiration, circulation, vomiting, swallowing, and pain control.
\subparagraph{reflex centers} for heart and circulation (vasomotor/cardiac) and
respiratory rhythmicity.

\paragraph{Cerebellum}
\begin{figure}[H]
	\centering
  	\includegraphics[width=\linewidth]{imgs/cerebellum.png}
	\caption{Cerebellum}
  	\label{fig:cerebellum}
\end{figure}

\paragraph{Spinal Cord}
\begin{figure}
	\centering
  	\includegraphics[width=\linewidth]{imgs/spinalCord.png}
	\caption{Spinal Cord}
  	\label{fig:spinalCord}
\end{figure}

The segmental organization of the spinal cord is ilustrated in Figure \ref{fig:spinalCord}. The spinal cord contains gray and white matter. 
The gray matter (inside part) of the spinal  cord consists of cell bodies of interneurons, motor neurons, and synaptic connections. Fibers of the motor neurons in the ventral horn leave the spinal cord to muscles (efferent/motor commands). Afferent/sensory axons enter through the dorsal horn and either synapse on sensory interneurons in the dorsal horn, or join the ascending tracts in the white matter.
The white matter of the spinal cord mostly consists of myelinated axons of motor and sensory neurons organized in columns (containing several fiber tracts) carrying information to (afferent/ascending) and from (efferent/descending) the brain.

\paragraph{Cranial Nerves}
Cranial nerves are the nerves that emerge directly from the brain (mostly from the
brainstem), in contrast to spinal nerves (which emerge from segments of the spinal cord).
Cranial nerves are generally named according to their structure or function. We have 12 cranial nerves: (i) olfactory, (ii) optical, (iii) oculomotor, (iv) trochlear, (v) trigeminal, (vi) abducens, (vii) facial, (viii) vestibulocochlear, (ix) glossopharyngeal, (x) vagus, (xi) accessory and (xii) hypoglossal nerve as we can see in Figure \ref{fig:cranialNerves}.
\begin{figure}
	\centering
  	\includegraphics[width=\linewidth]{imgs/cranialNerves.png}
	\caption{Cranial Nerves}
  	\label{fig:cranialNerves}
\end{figure}

The cranial nerves provide motor and sensory innervation mainly to the structures within the head and neck. The sensory innervation includes sensation such as temperature and touch, and innervation such as taste, vision, smell, balance and hearing. The vagus nerve (x) provides sensory and autonomic (parasympatheic) innervation to most of the organs in the chest and abdomen.

\paragraph{Meninges}
The meninges are the three membranes that envelop the brain and spinal cord. In mammals, the meninges are the \textbf{dura mater}, the \textbf{arachnoid mater}, and the \textbf{pia mater}. The inflamation of the meninges is called Meningitis.
\begin{itemize}
\item Dura mater: leather-like, inflexible layer surrounding the CNS and spinal cord. Inner and outer layers, containing large venous sinuses (e.g. superior sagittal sinus).
\item Arachnoid mater: loose connective tissue bridging the liquor-filled space (subarachnoidal space) between dura mater and pia mater. Contains all larger blood vessels.
\item Pia mater: translucent, thin membrane directly covering the entire surface of the brain, follows all sulci and gyri.
\end{itemize}

\paragraph{Ventricles and cerebrospinal fluid}
\begin{figure}
	\centering
  	\includegraphics[width=\linewidth]{imgs/ventricles-of-brain.png}
	\caption{Ventricles}
  	\label{fig:cranialNerves}
\end{figure}
The ventricles of the brain are a communicating network of cavities filled with cerebrospinal fluid (CSF) and located within the brain parenchyma. The ventricular system is composed of 2 lateral ventricles, the third ventricle, the cerebral aqueduct, and the fourth ventricle.
Some disorders on the ventricles cause diseases: neurodevelopmental (schizophrenia), neurodegenerative (alzheimer).

CSF is clear fluid, high content of NaCl, contains glucose and K+, low in proteins, very few cells (lymphocytes). It turnover three times a day. It flows throughout the ventricular system and is absorbed back into the bloodstream (via bloodbrain-barrier).
Cerebrospinal fluid is located in the subarachnoid space between the arachnoid mater and the pia mater.

\subparagraph{Functions of CSF} Buoyancy, Protection and Homeostasis. \\
\textbf{Buoyancy}: The actual mass of the human brain is approx. 1500 grams; however, the net weight of the brain suspended in the CSF is equivalent to a mass of 25 grams. The brain therefore exists in neutral buoyancy, which allows the brain to maintain its density without being impaired by its own weight, which would cut off blood supply. \\
\textbf{Protection}: CSF protects the brain tissue from injury when jolted or hit. In addition, it helps regulating intracranial pressure (lowering CSF production can help preventing brain ischemia). \\
\textbf{Homeostasis}: Through absorption back into the blood stream, CSF can rinse “metabolic waste” from the CNS, allowing for a homeostatic regulation of the brain. \\
Commont related pathology: hidrocephalus - abnormal accumulation of CSF within the brain. Can be congenital or acquired postnatally. Most common cause is aqueductal stenosis (passage between the 3rd and 4th ventricle is blocked or to narrow), so fluid accumulates in the upper ventricles.

\paragraph{Cerebral circulation}
The brain is one of the most metabolically active organs in the body! Uses approximately 20-25\% of the body’s total energy requirements (despite accounting for only 2\% of the body’s mass). The brain stores little energy as glycogen and relies mostly on circulating glucose. The rate of the cerebral blood flow in the adult is typically 750 milliliters per minute, representing 15\% of the cardiac output.

\subparagraph{Arteries} Supply oxygen-rich blood from heart to brain. Main branches of the internal carotids: anterior cerebral artery and middle cerebral artery. Main branches of the vertebral / basilar arteries: 3 arteries supplying the cerebellum and posterior cerebral artery.
\subparagraph{Veins} Carry oxigen-depleted blood away from brain

\begin{figure}
	\centering
  	\includegraphics[width=10cm]{imgs/avm_large.jpg}
	\caption{Arteries and Veins of the Brain}
  	\label{fig:arteriesVeinsBrain}
\end{figure}

\subsection{Comparative Neuroanatomy}

\subsubsection{Does brain size matter?}
\begin{figure}
	\centering
  	\includegraphics[width=10cm]{imgs/comparativeBrainSize.jpg}
	\caption{Brain size of different species}
  	\label{fig:comparativeBrainSizes}
\end{figure}

Is there a relationship between the size of an animal's brain and some kind of “behavioural complexity”? Not really. Elephants and whales have brains 4 to 5 times the size of a human being's, yet their behaviour is generally agreed to be less complex than ours.

\paragraph{Encephalization Quotient (EQ)} describes brain size as a ratio of the expected average brain size relative to the actual body weight. EQ of humans: $\approx$ 7.5 (Human brains are 7.5x bigger than what one would expect for species of this size.) EQ of sq. monkeys: $\approx$ 1.1.

\paragraph{Body mass and number of neurons} A capybara has 1,600,000,000 neurons and a common squirrel monkey (much smaller than a capybara) has 3,246,000,000 neurons.

\subsubsection{Brain evolution in view of cortical expansion}
Cortical expansion is often equated with "brain evolution”, whereby the relative size of the cerebral cortex increases while the relative size of the cerebellum remains fairly constant. We can see in Figure \ref{fig:corticalExpansion} the human cortical expansion is relative but does not affect each region simirlaly.

\begin{figure}
	\centering
  	\includegraphics[width=\linewidth]{imgs/comparativeNeuroanatomy.jpg}
	\caption{Cortical expansion}
  	\label{fig:corticalExpansion}
\end{figure}

\subsubsection{Cross-species comparison of cortical areas}

The human prefrontal cortex is responsible for planning, atention, working memory, cognitive flexibility and impulsivity. The human PFC is divided in dorsolateral PFC, anterior cingulate cortex and anterior PFC (or medial PFC). Rats (and mice) also have PFC, with similar responsabilities: lesions to the medial parte of PFC (mPfc) lead to working memory impairements as evident by the \textbf{increased number of working	memory errors in the 8-arm radial arm maze}.
\\
The rodent prefrontal cortex (PFC) is not as anatomically complex as the primate; however,
many of the critical neuroanatomical and functional characteristics are preserved in
rodents, which allow meaningful cross species comparisons relevant to study of the
neurocognitive and neurobiological mechanisms that underlie changes in executive
functioning across the lifespan. The medial portion of rodent PFC [which includes anterior cingulate (aCg), prelimbic (PL), and infralimbic (IL) cortices] shares strong anatomical homology with primate dorsolateral PFC

\subsubsection{Cross-species comparison of subcortical areas}
\begin{figure}[H]
	\centering
  	\includegraphics[width=\linewidth]{imgs/subcorticalAreas.png}
	\caption{Cross-species comparison of subcortical areas}
  	\label{fig:subcorticalAreas}
\end{figure}
	
\paragraph{Hippocampus}

\subparagraph{Hippocampal anatomy} The longitudinal axis of the hippocampus is described as ventrodorsal in rodents and as anteroposterior in primates. A rotation of 90-degree is required for the rat hippocampus to have the same orientation as that of primates, as you can see on Figure \ref{fig:hippocampusAnatomy}.
\begin{figure}[H]
	\centering
  	\includegraphics[width=\linewidth]{imgs/cross-species-hippocampus-anatomy.png}
	\caption{Cross-species hyppocampus anatomy}
  	\label{fig:hippocampusAnatomy}
\end{figure}

\subparagraph{Hippocampal Functions} In London taxi drivers were observed an increased brain activity associated with spatial navigation in the \textbf{right hippocampus} and left tail of the caudate. In rats the effect of hippocampal lesions on reference learning and memory was tested using the Morris water maze experiment. As bigger is the lesion on dorsal hippocampall, as bigger the deficit in the acquisition of spatial reference. Not so big deficit if the lesion were in the ventral hippocampal.

Experiment: The position of a submerged platform is constant from trial to trial at a given test day as well as form test day to test day. Animals are repeatedly placed into the tank with varying starting positions; with the help of spatial distal cues as reference points, they are required to find the invisible platform. Following completion of the acquisition phase, the platform is removed from the tank. The animals are once again placed in the tank; the critical measure here is whether the animals would “remember” the position of the platform and therefore would spent more time in quadrant where the platform was positioned before.

\paragraph{Amygdala}

\subparagraph{Amygdalar Anatomy} Primary amygdalar nuclei and basic circuit connections and
function are conserved across species. An enlarged image of the basolateral complex of the
amygdala (BLA) and central nucleus of the amygdala (CeA) or analogues are shown next to a coronalsection from the brains of a lizard, rat, cat, monkey, and human, in Figure \ref{fig:amygdalaAnatomy}.

\begin{figure}
	\centering
  	\includegraphics[width=\linewidth]{imgs/amygdalar-anatomy.png}
	\caption{Cross-species amygdalar anatomy}
  	\label{fig:amygdalaAnatomy}
\end{figure}

\subparagraph{Amygdalar Functions} In post-traumatic stress disorders (PTST), the amygdala is hyperactive in response to negative emotional stimulli vs. neutral and positive stimulli. In rodents the investigation of amygdalar function is tested using the \textbf{classical (pavlovian) fear conditioning}. In rats with amygdala lesions, the response to the non-threatening doesn't happen anymore.

Experiment: present a non-threatening stimulus (like a sound) with a noxius stimulus (like a midle shock) until the animal shows a fear response not just to the shock but also to the sound alone.

\paragraph{Basal ganglia}

\subsection{Exercises}

\paragraph{Coronal section - I}
\begin{figure}[H]
	\centering
  	\includegraphics[width=\linewidth]{imgs/coronal-section-I.png}
	\caption{Coronal section I}
  	\label{fig:coronalSectionI}
\end{figure}

\paragraph{Coronal section - II}
\begin{figure}[H]
	\centering
  	\includegraphics[width=\linewidth]{imgs/coronal-section-II.png}
	\caption{Coronal section II}
  	\label{fig:coronalSectionI}
\end{figure}

\paragraph{Coronal section - III}
\begin{figure}[H]
	\centering
  	\includegraphics[width=\linewidth]{imgs/coronal-section-III.png}
	\caption{Coronal section III}
  	\label{fig:coronalSectionI}
\end{figure}

\paragraph{Coronal section - IV}
\begin{figure}[H]
	\centering
  	\includegraphics[width=\linewidth]{imgs/coronal-section-IV.png}
	\caption{Coronal section IV}
  	\label{fig:coronalSectionI}
\end{figure}

\paragraph{Horizontal section}
\begin{figure}[H]
	\centering
  	\includegraphics[width=\linewidth]{imgs/horizontal-section.png}
	\caption{Horizontal section}
  	\label{fig:coronalSectionI}
\end{figure}



\section{Molecular \& Cellular Neuroscience}
\subsection{Building a central nervous system}
\subsection{Excitability}
\subsection{Glia and more}
\subsection{Synapses}

\section{Systems Neuroscience}
\subsection{Somatosensory and Motor Systems}
\subsection{Visual System}
\subsection{Auditory \& Vestibular System}
\subsection{Circuits underlying Emotion}
\subsection{Learning in artificial and biological neural networks}

\section{Answers}
\subsection{Human \& Comparative neuroanatomy}

\paragraph{Coronal section - I}
\begin{figure}[H]
	\centering
  	\includegraphics[width=\linewidth]{imgs/coronal-section-I-answer.png}
	\caption{Coronal section I}
  	\label{fig:coronalSectionI}
\end{figure}

\paragraph{Coronal section - II}
\begin{figure}[H]
	\centering
  	\includegraphics[width=\linewidth]{imgs/coronal-section-II-answer.png}
	\caption{Coronal section II}
  	\label{fig:coronalSectionI}
\end{figure}

\paragraph{Coronal section - III}
\begin{figure}[H]
	\centering
  	\includegraphics[width=\linewidth]{imgs/coronal-section-III-answer.png}
	\caption{Coronal section III}
  	\label{fig:coronalSectionI}
\end{figure}

\paragraph{Coronal section - IV}
\begin{figure}[H]
	\centering
  	\includegraphics[width=\linewidth]{imgs/coronal-section-IV-answer.png}
	\caption{Coronal section IV}
  	\label{fig:coronalSectionI}
\end{figure}

\paragraph{Horizontal section}
\begin{figure}[H]
	\centering
  	\includegraphics[width=\linewidth]{imgs/horizontal-section-answer.png}
	\caption{Horizontal section}
  	\label{fig:coronalSectionI}
\end{figure}


\newpage

\section{Previous Exams}
Note this answers were provided by students and were not verified by a teacher. Use them at your own risk.

\subsection{2011}

\paragraph{Q1. Discuss the functions and structures of the hypothalamus as
discussed in the lecture material. Label 18 structures in 2 different coronal slices}
\paragraph{2. Developement: Describe how DRG sensory neurons development in comparison to motor neurons. How are cell boundaries formed in general and among the specific motor/sensory nerves}
\paragraph{Q3. Axon Guidance: what were sperry's findings that support the chemoaffinity
hypothesis. What molecules are involved in this and how do they function.}
\paragraph{Q4. Describe from how sound is encoded neurally (from entering the ear
to being perceived as sound in brain - complete pathway)}
\paragraph{Q5. Draw a flowchart for a typical neuroproteomics experiment}
\paragraph{Q6. Fill in the blank and multiple choice questions from Tobi's lecture: Who invented the term Neuro Engeneering? What is CMOS? Power consumption of brain. Synchronous logic is ubiquitous slide know physiologists friend photodiodes - how they are similar to retina CARVER MEAD}


\subsection{2010}
\paragraph{Q1. Auditory pathway}
\paragraph{Q2. Development of CNS and PNS}
\paragraph{Q3. Boundary building (one slide, different cell type)}
\paragraph{Q4. Pathfinding (Chemoaffinity, give 2 examples)}
\paragraph{Q5. Anatomy (hypothalamus, position and function)}
\paragraph{Q6. Neuromorphic engineering}

\subsection{2009}
\paragraph{Q1. Neuroanatomy: which of the 12 cranial nerves origin and/or end in the brainstem? What are their respective sensory, motor and /or vegetative functions ?(please describe in detail) Which nuclei of the cranial nerves are located in the mesencephalon?}
\paragraph{Q2. Auditory system: Describe differences between "conductive hearing loss" and "sensorineural hearing loss". Describe the classical test which is often used to determine between both forms of hearing loss. Describe biological causes and current tratments aids for such hearing impairments.}
\paragraph{Q3. Proteomics in neuroscience:}
\subparagraph{a. explain the term "proteome"}
\subparagraph{b. what are the benefits of measuring the proteome in addition to the genome?}
\subparagraph{c. Describe what a mass spectrometry is doing in principle.}
\subparagraph{d. How would you quantify proteins in a proteomic experiment? Please name and describe at least 2 proteomics technologies}
\subparagraph{e. Why is the proteome more complex compared to genome? Name and describe 3 reasons.}
\paragraph{Q4. Ion channels: What are the principal functions of dendrites, axon and nerves endings in the transcription of signals through the nervous system? Which types of ion channels are critical for the function of each of these 3 structures? Provide specific examples.}
\paragraph{Q5. Neural network: Explain the temporal and spatial network definition. Give an example for each network definition and describe how you can detect these networks in the brain.}
\paragraph{Q6. Neuromorphic engineering: Considering organizing principles used in biological retina explain (...)}

\subsection{2008?}
\paragraph{Q1. Describe the diencephalon and its major components according to the text "the brain in a nutshell"}
\paragraph{Q2. Compare structure and development of the cerebellum and the cortex}
\paragraph{Q3. What evidence did Sperry find that supports his chemoaffinnity hypothesis? Have Sperrys proposed "recognition molecules" been found? If yes name one example and describe what properties of this molecyles support its role as a recognition molecule}
\paragraph{Q4. Describe the structure of a voltage potassium channel. Explain the mechanisms that make the channel selective for only potassium ions.}
\paragraph{Q5. Describe three functional properties of neurons in v1 that are absent in the LGN. For each property describe in detail an experiment that illustrates it including the type of stimulus and the observed neural responses. Finally, choose one of these three properties and explain as presisely as possible how it can emerg at the cortical level.}
\paragraph{Q6. Which dynamic processes occur in single neuron and the local neural circuit during signal flow through a neural network? Name critical structural and functional aspects and discuss how they can be measured experimentally}

\subsection{2007}

\paragraph{Q1. Label each part of the brain, two coronal section, 18 areas. Describe the lobes of cortex, according to the handout}
\paragraph{Q2. Development. Compare the cell migration to form the cortex and the migration in the peripheral neural system forming…}
\paragraph{Q3. Development. About neurotrophic factor. What’s the experiment led to the finding of neurotrophic factor? Compare trophic and tropic factor.}
\paragraph{Q4. What’s the difference between ionotropic and metabotropic receptors?}
\paragraph{Q5. serotonin}
\paragraph{Q6. insect eye}

\subsection{2006}
\paragraph{Q1. Label each part of the brain.Describe the components of midbrain according to the description in the shells of the brain}
\paragraph{Q2. Why the ion channel is selective to a ion with particular polarity, e.g. why the positive channel allows cations going through? List different ion channels in different part of the neuron: (1) axon, (2) axon terminal and (3) postsynaptic membrane}
\paragraph{Q3. How does cells differentiate in neural tube? How does cells differentiate in peripheral nervous system?}
\paragraph{Q4. What is the structure and function of myelinated nerve in PNS?}
\paragraph{Q5. Describe the structure and function of dopaminergic systems, the relation between the systems and psychology and neuropharmacology}
\paragraph{Q6. Visual system: describe the two columns in V1, and give out the definitions of them. What is the relation between this two columns and also between them and other part of visual system. Give out one experiment for testing of each column.}


\subsection{2005}
\paragraph{Q1. Draw the connectivity between motor cortex, thalamus, basal ganglia and cerebellum for motor control and show which connections are excitatory or inhibitory - Coronal view to fill with 16 names}

\paragraph{Q2. Model organisms for understanding human brain development and function. Give three general advantages of these models. Compare in a table with advantages and disadvantages the models: Drosophila, C. Elegans, Zebrafish, Mouse. Give an example for each of how it helped understand the brain}

\paragraph{Q3. Two main route for cell death, detail differences}

\paragraph{Q4. Myelin: structure and function}

\paragraph{Q5. What are the two main modes of electrophysiological communication between neurons? Describe structure.}

\paragraph{Q6. Vision:}
\subparagraph{A. cites six roles for vision in insects}
\subparagraph{B. 1. what structural/functional differences between insect and human eyes 2. in what experiments are insects eyes worse or better than human eyes? 3. five share important features of insect and human eyes}
\subparagraph{C. 1. what is flow field? 2. draw the flow field perceived by a fly flying straight in a long corridor 3. what is ... field? ( i forgot the name oops) 4. draw pure rotation force field and its matched field filter}


\subsection{2004}
Some questions (especially number 6) are about subject not taught anymore
during the first ZNZ introduction semester, so don't worry about them.

\paragraph{Q1. Describe the major formations involving the hippocampus in the associative cortex. Coronal slices, 16 areas to be labelled}
\paragraph{Q2. Structure and role of myelinating cells in the adult nervous system.}
\paragraph{Q3. Name some crucial functions of Neurotrophic factors.}
\paragraph{Q4. How is information transported in the nervous system? Explain features and function.}
\paragraph{Q5. In verterbrates the vision system has some special wiring pattern. What's special about it (as in, how is it different to olfaction)? Explain biological/physiological means in the development of vision.}
\paragraph{Q6. Imagine year 2020. Human genomics has advanced to the point where you not only can choose the gender and hair color of your child, but also apply specific changes to the visual system. Name 6 changes to the human visual system you would apply to your kid. Explain why you chose them and what physiological implications they would have.}

\subsection{All Question - topics}
\subsubsection{Cytology}
\paragraph{Q1. What is the structure and function of a myelinated peripheral nerve?}
\paragraph{Q2. Myelin : structure and function}
\paragraph{Q3. Structure and role of myelinating cells in the adult nervous system.}

\subsubsection{Anatomy}
\paragraph{Q1. Describe the major formations involving the hippocampus in the associative cortex.}
\paragraph{Q2. Label 2 coronal slices, 16 areas to be labelled (twice)}
\paragraph{Q3. Mesencephalon: components \& nuclei (brain in a nutshell)}
\paragraph{Q4. Motor activity structures and fibres}
\paragraph{Q5. Output structures and structures modulating output}
\paragraph{Q6. Draw the connectivity between motor cortex, thalamus, basal ganglia and cerebellum for motor control and show which connections are excitatory or inhibitory}
\paragraph{Q8. Describe the diencephalon and ist major components}

\subsubsection{Development}
\paragraph{Q1. Name some crucial functions of Neurotrophic factors.}
\paragraph{Q2. How do different types of neurons differentiate in the neural tube?}
\paragraph{Q3. How do different types of neurons differentiate in the periphery?}
\paragraph{Q4. Example for migration}
\paragraph{Q5. Compare in a table with advantages and disadvantages the models: Drosophila, C. Elegans, Zebrafish, Mouse. Give an example for each of how it helped understand the brain}
\paragraph{Q6. Compare structure and development of the cerebellum and the cortex}
\paragraph{Q7. What evidence did Sperry find that supports his chemoaffinity hypothesis?}
\paragraph{Q8. Have Sperry’s proposed recognition molecules been found? If yes, name one example and describe what properties of this molecule supports it’s role as a recognition molecule.}

\subsubsection{Synaptogenesis}
\paragraph{Q1. Two main routes for cell death, detail differences}

\subsubsection{Ion Channel}
\paragraph{Q1. Why is the ion channel selective to an ion with particular polarity, e.g. why the positive channel allows cations going through? List different ion channels in different part of the neuron: (1) axon, (2) axon terminal and (3) postsynaptic membrane}
\paragraph{Q2. Describe the structure of a voltage-gated potassium channel. Explain the mechanisms that make the channel selective for only potassium ions.}

\subsubsection{Synaptic Transmission}
\paragraph{Q1. How is information transported in the nervous system? Explain features and function.}
\paragraph{Q2. Structural selectivity feature allowing cationic channel to conduct positive but not negative ions}
\paragraph{Q3. Transmission of neuronal signals -> list ion channels critical for i) Axons ii) Dendrites -> difference excitatory, inhibitory and iii) Nerve Terminals}
\paragraph{Q4. Function of AP, resting potential, synaptic potential channels for all these potentials}
\paragraph{Q5. What are the two main modes of electrophysiological communication between neurons? Describe structure.}

\subsubsection{Visual System}
\paragraph{Q1. Describe the two columns in V1, and give out the definitions of them. What is the relation between this two columns and also between them and other part of visual system. give out one experiment for testing of each column.}
\paragraph{Q2. Describe three functional properties of neurons in area V1 that are absent in the Lateral Geniculate Nucleus. For each property describe in detail an experiment that illustrates it, including the type of stimulus and the observed neuronal responses. Finally, choose one these three properties and explain as precisely as possible how it can emerge at the cortical level.}
\paragraph{Q3. In verterbrates the vision system has some special wiring pattern.
What's special about it (as in, how is it different to olfaction)? Explain biological/physiological meansin the development of vision.}
\paragraph{Q4. Imagine year 2020. Human genomics has advanced to the point where you
not only can choose the gender and hair color of your child, but also apply specific
changes to the visual system. Name 6 changes to the human visual system you would apply to your kid. Explain why you chose them and what physiological implications they would have.}

\subsubsection{Neuromorphic engineering}

\paragraph{Q1. Differences in organising principle between electronic and neural computation}
\paragraph{Q2. Illustrate neural computation principles by specific examples (e.g. retina), explain functional utility}

\subsubsection{Neural networks in vivo}
\paragraph{Q1. Which dynamic process occur in single neurons and the local neural circuit during signal flow through a neuronal network? Name critical structural and functional aspects and discuss how they can be measured experimentally.}

\subsubsection{Neural computation}
\paragraph{Q1. Fill in the blank and multiple choice questions from Tobi's lecture: Who invented the term Neuro Engeneering? What is CMOS? Power consumption of brain. Synchronous logic is ubiquitous slide know physiologists friend photodiodes - how they are similar to retina CARVER MEAD}

\newpage

\section{References}
The pictures used in this summary are from the class slide sets and belong to their respective owners. In the context of the summary they are
used for educational purposes only.
\end{document}
