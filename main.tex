\documentclass[12pt,article,oneside,a4paper]{memoir}

%% Packages
%% ========
\usepackage{graphicx}
\usepackage{titlesec}
\usepackage{wrapfig}
	
\setcounter{secnumdepth}{4}

\titleformat{\paragraph}
{\normalfont\normalsize\bfseries}{\theparagraph}{1em}{}
\titlespacing*{\paragraph}
{0pt}{3.25ex plus 1ex minus .2ex}{1.5ex plus .2ex}

%% many common packages
\input{commonpackages}

%% Some more packages that you may want to use.  Have a look at the
%% file, and consult the package docs for each.
\input{extrapackages}

%% Our layout configuration.
\input{layoutsetup}

%% Theorem environments.  You will have to adapt this for a German
%% thesis.
\input{theoremsetup}

%% Helpful macros.
\input{macrosetup}

%%page layout settings and listing templates etc.
\input{settings}

\title{\textbf{ZNZ HS16 Introduction to Neuroscience I} \\
       Fall 2016\\\normalsize version 1.0}

\author{
	Vanessa Leite
	\vspace{2em}
	\\Repository page: \url{https://github.com/ssinhaleite/znz-intro-to-neuroscience-I-summary}\\
	Contact \href{mailto:vrcleite@gmail.com}{vrcleite@gmail.com} if you have any questions.}
	\thesistype{The Summary of the lectures in 2016}
	\department{ZNZ - Institute of Neuroinformatics, ETH}
	\date{\today}

\begin{document}
\frontmatter


%% DO NOT CHANGE.
\begin{titlingpage}
  \calccentering{\unitlength}
  \begin{adjustwidth*}{\unitlength-24pt}{-\unitlength-24pt}
    \maketitle
  \end{adjustwidth*}
\end{titlingpage}

\mainmatter

%% This change is needed if the article option for the memoir document class
%% is used, in order to count sections (article) as if they were chapters (memoir)
\counterwithout{section}{chapter}

%% Our content

\newpage
\clearpage
\pagenumbering{roman}
\setcounter{tocdepth}{3}
\setcounter{secnumdepth}{2}
\tableofcontents

\clearpage
\pagenumbering{arabic}

\newpage

\section{Human \& Comparative Neuroanatomy}
\subsection{Human Neuroanatomy}
\subsubsection{Why do we need to know the brain}
The famous case of the HM pacient: Henry Gustav Molaison went through a surgery on brain to cure his epilepsy. However, during the surgery two holes were drilled in the front of his skull and a portion of his brain, the front half of the hippocampus on both sides, and most of the almond-shaped amygdala, was sucked out. The procedure went badly wrong and Henry, then aged 27, was left with no ability to store or retrieve new experiences. He lived the subsequent 55 years of his life, until his death in 2008, in the permanent present moment.

\subsubsection{Nervous system}
The nervous system is divided in two parts: the Central Nervous System (CNS) and
the Peripheral Nervous System (PNS). Each part has its own divisions as we can see in Figure \ref{fig:nervousSystem}.

\begin{figure}
  \includegraphics[width=\linewidth]{imgs/division_nervous_system.jpg}
  \caption{Division of the Nervous System}
  \label{fig:nervousSystem}
\end{figure}

\begin{itemize}
\item CNS
\subitem Brain
\subitem Spinal Cord
\item PNS
\subitem Somatic and autonomic nervous system
\end{itemize}

Both system contains gray and white matter.
In the PNS the gray matter contains \textbf{ganglia}: collection of neuron cell
bodies -, the white matter contains \textbf{nerves}: bundles of axons.
In the CNS the gray matter is divided in:
\begin{itemize}
\item Neural cortex - gray matter on the surface of the brain
\item Nuclei - collection of neuron cell bodies in the interior of CNS
\item Centers - collection of neuron cell bodies in CNS, each center has specific processing functions
\item High centers - the most complex centers in brain.
\end{itemize}
The white matter in CNS is divided in two parts: the \textbf{tracts or fasciculus}: bundle of CNS axons
that share a common origin and destination -, and the \textbf{columns or funiculus}: several tracts (fasciculi) that form an anatomically distinct mass

The centers and tracts that connect the brain with other organs and system in
the body are called \textbf{pathways}. The ascending (sensory) pathway is called afferent.
The descending (motor) pathway is called efferent.

Figure \ref{fig:brain} shows the macro division of the brain: Telencephalon, Diencephalon, Brain stem (Midbrain or Mesencephalon, Pons and Cerebellum and Medulla oblongata) and Medulla spinallis), in \ref{fig:brain2} we can see some views of the brain.
Also part of the anatomy of the brain: cranial nerves, meninges, ventricles / cerebrospinal fluid and cerebral circulation.

\begin{figure}
  \includegraphics[width=\linewidth]{imgs/viewsOfTheBrain.png}
  \caption{Division of the brain}
  \label{fig:brain}
\end{figure}

\begin{figure}
  \includegraphics[width=\linewidth]{imgs/viewsOfTheBrain2.png}
  \caption{Views of the Brain}
  \label{fig:brain2}
\end{figure}

\paragraph{Telencephalon - or Forebrain}
The telencephalon (the biggest part of the brain) is divided in lobes, functional cortical areas, basal ganglia and limbic system.

The four lobes (frontal, occitopital, temporal and parietal) are presented in Figure \ref{fig:boundaries}.

\subparagraph{Gray matter}
The macroscopic boundaries of the gray matter are Gyri, Sulci and Commissural fiber tracts. Each one is divided as follows:
\begin{itemize}
\item Gyri
\subitem precentral gyrus
\subitem postcentral gyrus
\subitem pars triangularis
\subitem angular gyrus
\subitem cingulate gyrus
\subitem parahippocampal gyrus
\item Sulci
\subitem central sulcus
\subitem lateral fissure
\subitem parieto-occipital sulcus
\subitem calcarine sulcus
\item Commissural fiber tracts
\subitem corpus callosum
\subsubitem Rostrum
\subsubitem Genu
\subsubitem Truncus
\subsubitem Splenium
\subitem anterior commissure
\end{itemize}

Figure \ref{fig:boundaries} shows the macroscopic boundaries of the gray matter. Besides the anatomical division, there is a functional division of the brain, where each area in the cerebral cortex has specific functional activities. The Wernicke's (language comprehension) and Broca's (speech production) areas are highlited in Figure \ref{fig:functionalBoundaries}.

In 1909 Korbinian Brodmann described areas of the cerebral cortex on the basis of
cytoarchitectural criteria. Areas differ in celltypes, layering and cell distribution, resulting in 52 Brodman Areas.

The human brain is gyrencephalic, i.e, is formed by giri, as the elephant brain. However other species can be  lissencephalic (the brain is smooth, without giri) as the domestic rabbit and the house mouse. Defects in the neuronal migration during early to mid gestation (12th to 24th weeks) leading to impaired development of gyri and sulci.

\begin{figure}
  \includegraphics[width=\linewidth]{imgs/macroscopic_boundaries.png}
  \caption{Macroscopic Boundaries - gray matter of the cortex}
  \label{fig:boundaries}
\end{figure}

\begin{figure}
  \centering
  \includegraphics[width=12cm]{imgs/functional_boundaries.png}
  \caption{Functional Division - gray matter of the cortex}
  \label{fig:functionalBoundaries}
\end{figure}

\newpage

\subparagraph{White matter}

\begin{wrapfigure}[11]{r}{0.5\textwidth}
	\centering
  	\includegraphics[width=5cm]{imgs/macroscopic_whiteMatter.png}
	\caption{White matter - macroscopic fibers}
  	\label{fig:macroscopic_whiteMatter}
\end{wrapfigure}

The white matter can be divided macroscopically and microscopically. Macroscopically we talk about fibers and microscopically we talk about cells. Figure \ref{fig:macroscopic_whiteMatter} exemplify the macroscopic division and Figure \ref{fig:microscopic_whiteMatter} exemplify the microscopic division, where we can see microglias, astrocyte and oligodendrocytes cells.

\begin{itemize}
\item Comissural fibers (red): link areas between the two hemispheres (corpus callosum, anterior commissure, posterior commissure)
\item Association fibers (green): link cortical areas of the same hemisphere.
\item Projecting fibers (blue): link the cortex with subcortical areas of the brain and the spinal cord.
\end{itemize}

\begin{figure}
  \centering
  \includegraphics[width=12cm]{imgs/microscopic_whiteMatter.png}
  \caption{White Matter - microscopic structures}
  \label{fig:microscopic_whiteMatter}
\end{figure}

\subparagraph{Basal Ganglia}

The basal ganglia are the principal subcortical components of a family of neuronal circuits which link the thalamus and cerebral cortex. It is crucial for the initiation and modulation of voluntary movement by sending their output to the motor cortex via the thalamus. In addition, the basal ganglia also contribute to a variety of behavioral and cognitive functions other than voluntary movement.

The basal ganglia is divided in:
\begin{itemize}
\item Striatum: is the major recipient of inputs from the substantia nigra, cerebral cortex, thalamus, and brain stem. In humans (and most primates) consist of the caudate nucleus, the putamen, and the nucleus accumbens. In rats and mices consist of caudate putamen (human caudate nucleus + putamen) and nucleus accumbens.
\item Globus Palidus: is divided into external and internal segments.
\subitem The internal segment (GPi) sends projections to the thalamus and pedunculopontine nucleus (a group of cells located in the brain stem).
\subitem The external segment (GPe) sends projections to the internal segment of the globus pallidus and to the subthalamic nucleus.
\item Susbtantia Nigra: is a midbrain (mesencephalon) structure and contains a dense population of dopamine cells. The substantia nigra can be subdivided into substantia nigra pars compacta and pars reticulata.
\end{itemize}

One of the disorders of the basal ganglia is the parkinson's disease, where the dopaminergic cells in the substantia nigra pars compacta are lost, it impairs motor skills, speech and other functions.

\subparagraph{Limbic system}
Divided in cingulate gyrus (superior portion of limbic lobe), parahippocampal gyrus (inferior portion of limbic lobe), hippocampus and amigdalar complex.
In Alzheimer's disease, the hippocampus is one of the first regions of the brain to suffer damage. memory problems (especially spatial memories) and disorientation appear among the first symptoms. People with extensive, bilateral hippocampal damage (such as in patients with progressed AD) may experience anterograde amnesia (the inability to form or retain new memories).
The amigdala is envolved in emotions.

\paragraph{Diencephalon}
The diencephalon is divided in thalamus, hypothalamus, epithalamus, subthalamus

\subparagraph{Thalamus}
The thalamus is the gatekeeper of the brain: it is important for the transfer of information from the periphery to sensory processing regions in the telencephalon. It has important gating (filtering) functions: it determines whether sensory information reaches conscious awareness in the neocortex and participates in the integration of motor information from the cerebellum and basal ganglia and transmits this information to cerebral areas concerned with movement.

\subparagraph{Hypothalamus}
The hypothalamus regulates several behaviors that are essential for homeostasis
and reproduction: growth, eating, drinking and maternal behavior, by regulating hormonal secretions from the pituitary gland. It is an important control center for the autonomic nervous system and for the hypothalamus-pituitary-adrenal (HPA) stress-response system.

\subparagraph{Neuroendocrinollogy of Hypothalamus}
\begin{enumerate}
\item Hypothalamus produces releasing hormones (rh) and inhibiting hormones (ih) that directly influence anterior pituitary hormone secretion.
\item Hypothalamus produces two hormones (oxytocin and antidiuretic hormone) that are stored in the posterior pituitary.
\item Hupothalamus overseesthe ANS (autonomic nervous system)thereby helping to stimulate the adrenal medulla via sympathetic innervation.
\end{enumerate}

\subparagraph{Epithalamus}
epithelial roof of the third ventricle, habenula, pineal body and afferent/efferent connections. It is responsible for the secretion of melatonin, regulation of day-night cycles, information processing related to olfaction.

\subparagraph{Subthalamus}
It is the continuation of the tegmentum. Functionally part of the basal ganglia (motor control).

\newpage

\paragraph{Mesencephalon - or Midbrain}
\begin{wrapfigure}[11]{r}{0.5\textwidth}
	\centering
  	\includegraphics[width=7cm]{imgs/mesencephalon.png}
	\caption{Mesencephalon - Functional units}
  	\label{fig:mesencephalon}
\end{wrapfigure}

The midbrain is a portion of the CNS associated with vision, hearing, motor control, sleep/wake, arousal (alertness), and temperature regulation. It comprises the tectum (or corpora quadrigemina), tegmentum, the cerebral aqueduct (or ventricular mesocoelia or "iter"), and the cerebral peduncles, as well as several nuclei and fasciculi. Caudally the midbrain adjoins the metencephalon (afterbrain) (pons and cerebellum). while rostrally it adjoins the diencephalon (thalamus, hypothalamus, etc). In Figure  \ref{fig:mesencephalon} the parts of the midbrain are listed.

\begin{enumerate}
\item Tectum (roof)
\subitem superior colliculus: visual and occulomotor reflexes
\subitem inferior colliculus: relay auditory tract
\item Tegmentum (floor)
\item Reticular formation: automatic processing of incoming sensation and outgoing motor
commands, helps to maintain consciousness, can initiate motor response to stimuli (see also
medulla oblongata!)
\item Red nucleus: involuntary control of background muscle tone and limb posture
\item Substantia nigra: regulates activity in the basal nuclei, degeneration of dopaminergic cells causes Parkinson’s disease
\item Cerebral peduncles: connect primary motor cortex with motor neurons in brain and spinal cord, carry ascending sensory information to thalamus
\item Ventral tegmental area (VTA): part of the limbic system, projects e.g. to nucleus accumbens and amygdala, emotional reinforcement.
\end{enumerate}

\paragraph{Pons}
\begin{wrapfigure}[7]{r}{0.5\textwidth}
	\centering
  	\includegraphics[width=7cm]{imgs/pons.png}
	\caption{Pons}
  	\label{fig:mesencephalon}
\end{wrapfigure}

Divded in two parts: locus coeruleus and pontine nuclei. The locus coeruleus (or blue spot) contains noradrenergic cells innervating large portions of the brain, mediating physiological response to panic and stress. The pontine nuclei receive fibers from all cortical areas and relay to the contralateral cerebellum.

\paragraph{Medulla oblongata}
\begin{figure}
	\centering
  	\includegraphics[width=\linewidth]{imgs/medullaOblongata.png}
	\caption{Medulla Oblongata}
  	\label{fig:medullaOblongata}
\end{figure}

It contains four main parts: olives, pyramid, reticular formation and reflex centers.\subparagraph{olive} relay nucleus for afferent connection from motor cortex and red nucleus, efferent to contralateral cerebellum.
\subparagraph{pyramid} contains descending cortico-spinal fibers.
\subparagraph{reticular formation (entire brain stem!)} containing the raphe nuclei and magno/parvocellular nuclei, which regulate respiration, circulation, vomiting, swallowing, and pain control.
\subparagraph{reflex centers} for heart and circulation (vasomotor/cardiac) and
respiratory rhythmicity.

\paragraph{Cerebellum}
\begin{figure}[H]
	\centering
  	\includegraphics[width=\linewidth]{imgs/cerebellum.png}
	\caption{Cerebellum}
  	\label{fig:cerebellum}
\end{figure}

\paragraph{Spinal Cord}
\begin{figure}
	\centering
  	\includegraphics[width=\linewidth]{imgs/spinalCord.png}
	\caption{Spinal Cord}
  	\label{fig:spinalCord}
\end{figure}

The segmental organization of the spinal cord is ilustrated in Figure \ref{fig:spinalCord}. The spinal cord contains gray and white matter. 
The gray matter (inside part) of the spinal  cord consists of cell bodies of interneurons, motor neurons, and synaptic connections. Fibers of the motor neurons in the ventral horn leave the spinal cord to muscles (efferent/motor commands). Afferent/sensory axons enter through the dorsal horn and either synapse on sensory interneurons in the dorsal horn, or join the ascending tracts in the white matter.
The white matter of the spinal cord mostly consists of myelinated axons of motor and sensory neurons organized in columns (containing several fiber tracts) carrying information to (afferent/ascending) and from (efferent/descending) the brain.

\paragraph{Cranial Nerves}
Cranial nerves are the nerves that emerge directly from the brain (mostly from the
brainstem), in contrast to spinal nerves (which emerge from segments of the spinal cord).
Cranial nerves are generally named according to their structure or function. We have 12 cranial nerves: (i) olfactory, (ii) optical, (iii) oculomotor, (iv) trochlear, (v) trigeminal, (vi) abducens, (vii) facial, (viii) vestibulocochlear, (ix) glossopharyngeal, (x) vagus, (xi) accessory and (xii) hypoglossal nerve as we can see in Figure \ref{fig:cranialNerves}.
\begin{figure}
	\centering
  	\includegraphics[width=\linewidth]{imgs/cranialNerves.png}
	\caption{Cranial Nerves}
  	\label{fig:cranialNerves}
\end{figure}

The cranial nerves provide motor and sensory innervation mainly to the structures within the head and neck. The sensory innervation includes sensation such as temperature and touch, and innervation such as taste, vision, smell, balance and hearing. The vagus nerve (x) provides sensory and autonomic (parasympatheic) innervation to most of the organs in the chest and abdomen.

\paragraph{Meninges}
The meninges are the three membranes that envelop the brain and spinal cord. In mammals, the meninges are the \textbf{dura mater}, the \textbf{arachnoid mater}, and the \textbf{pia mater}. The inflamation of the meninges is called Meningitis.
\begin{itemize}
\item Dura mater: leather-like, inflexible layer surrounding the CNS and spinal cord. Inner and outer layers, containing large venous sinuses (e.g. superior sagittal sinus).
\item Arachnoid mater: loose connective tissue bridging the liquor-filled space (subarachnoidal space) between dura mater and pia mater. Contains all larger blood vessels.
\item Pia mater: translucent, thin membrane directly covering the entire surface of the brain, follows all sulci and gyri.
\end{itemize}

\paragraph{Ventricles and cerebrospinal fluid}
\begin{figure}
	\centering
  	\includegraphics[width=\linewidth]{imgs/ventricles-of-brain.png}
	\caption{Ventricles}
  	\label{fig:cranialNerves}
\end{figure}
The ventricles of the brain are a communicating network of cavities filled with cerebrospinal fluid (CSF) and located within the brain parenchyma. The ventricular system is composed of 2 lateral ventricles, the third ventricle, the cerebral aqueduct, and the fourth ventricle.
Some disorders on the ventricles cause diseases: neurodevelopmental (schizophrenia), neurodegenerative (alzheimer).

CSF is clear fluid, high content of NaCl, contains glucose and K+, low in proteins, very few cells (lymphocytes). It turnover three times a day. It flows throughout the ventricular system and is absorbed back into the bloodstream (via bloodbrain-barrier).
Cerebrospinal fluid is located in the subarachnoid space between the arachnoid mater and the pia mater.

\subparagraph{Functions of CSF} Buoyancy, Protection and Homeostasis. \\
\textbf{Buoyancy}: The actual mass of the human brain is approx. 1500 grams; however, the net weight of the brain suspended in the CSF is equivalent to a mass of 25 grams. The brain therefore exists in neutral buoyancy, which allows the brain to maintain its density without being impaired by its own weight, which would cut off blood supply. \\
\textbf{Protection}: CSF protects the brain tissue from injury when jolted or hit. In addition, it helps regulating intracranial pressure (lowering CSF production can help preventing brain ischemia). \\
\textbf{Homeostasis}: Through absorption back into the blood stream, CSF can rinse “metabolic waste” from the CNS, allowing for a homeostatic regulation of the brain. \\
Commont related pathology: hidrocephalus - abnormal accumulation of CSF within the brain. Can be congenital or acquired postnatally. Most common cause is aqueductal stenosis (passage between the 3rd and 4th ventricle is blocked or to narrow), so fluid accumulates in the upper ventricles.

\paragraph{Cerebral circulation}
The brain is one of the most metabolically active organs in the body! Uses approximately 20-25\% of the body’s total energy requirements (despite accounting for only 2\% of the body’s mass). The brain stores little energy as glycogen and relies mostly on circulating glucose. The rate of the cerebral blood flow in the adult is typically 750 milliliters per minute, representing 15\% of the cardiac output.

\subparagraph{Arteries} Supply oxygen-rich blood from heart to brain. Main branches of the internal carotids: anterior cerebral artery and middle cerebral artery. Main branches of the vertebral / basilar arteries: 3 arteries supplying the cerebellum and posterior cerebral artery.
\subparagraph{Veins} Carry oxigen-depleted blood away from brain

\begin{figure}
	\centering
  	\includegraphics[width=10cm]{imgs/avm_large.jpg}
	\caption{Arteries and Veins of the Brain}
  	\label{fig:arteriesVeinsBrain}
\end{figure}

\subsection{Comparative Neuroanatomy}

\subsubsection{Does brain size matter?}
\begin{figure}
	\centering
  	\includegraphics[width=10cm]{imgs/comparativeBrainSize.jpg}
	\caption{Brain size of different species}
  	\label{fig:comparativeBrainSizes}
\end{figure}

Is there a relationship between the size of an animal's brain and some kind of “behavioural complexity”? Not really. Elephants and whales have brains 4 to 5 times the size of a human being's, yet their behaviour is generally agreed to be less complex than ours.

\paragraph{Encephalization Quotient (EQ)} describes brain size as a ratio of the expected average brain size relative to the actual body weight. EQ of humans: $\approx$ 7.5 (Human brains are 7.5x bigger than what one would expect for species of this size.) EQ of sq. monkeys: $\approx$ 1.1.

\paragraph{Body mass and number of neurons} A capybara has 1,600,000,000 neurons and a common squirrel monkey (much smaller than a capybara) has 3,246,000,000 neurons.

\subsubsection{Brain evolution in view of cortical expansion}
Cortical expansion is often equated with "brain evolution”, whereby the relative size of the cerebral cortex increases while the relative size of the cerebellum remains fairly constant. We can see in Figure \ref{fig:corticalExpansion} the human cortical expansion is relative but does not affect each region simirlaly.

\begin{figure}
	\centering
  	\includegraphics[width=\linewidth]{imgs/comparativeNeuroanatomy.jpg}
	\caption{Cortical expansion}
  	\label{fig:corticalExpansion}
\end{figure}

\subsubsection{Cross-species comparison of cortical areas}

The human prefrontal cortex is responsible for planning, atention, working memory, cognitive flexibility and impulsivity. The human PFC is divided in dorsolateral PFC, anterior cingulate cortex and anterior PFC (or medial PFC). Rats (and mice) also have PFC, with similar responsabilities: lesions to the medial parte of PFC (mPfc) lead to working memory impairements as evident by the \textbf{increased number of working	memory errors in the 8-arm radial arm maze}.
\\
The rodent prefrontal cortex (PFC) is not as anatomically complex as the primate; however,
many of the critical neuroanatomical and functional characteristics are preserved in
rodents, which allow meaningful cross species comparisons relevant to study of the
neurocognitive and neurobiological mechanisms that underlie changes in executive
functioning across the lifespan. The medial portion of rodent PFC [which includes anterior cingulate (aCg), prelimbic (PL), and infralimbic (IL) cortices] shares strong anatomical homology with primate dorsolateral PFC

\subsubsection{Cross-species comparison of subcortical areas}
\begin{figure}[H]
	\centering
  	\includegraphics[width=\linewidth]{imgs/subcorticalAreas.png}
	\caption{Cross-species comparison of subcortical areas}
  	\label{fig:subcorticalAreas}
\end{figure}
	
\paragraph{Hippocampus}

\subparagraph{Hippocampal anatomy} The longitudinal axis of the hippocampus is described as ventrodorsal in rodents and as anteroposterior in primates. A rotation of 90-degree is required for the rat hippocampus to have the same orientation as that of primates, as you can see on Figure \ref{fig:hippocampusAnatomy}.
\begin{figure}[H]
	\centering
  	\includegraphics[width=\linewidth]{imgs/cross-species-hippocampus-anatomy.png}
	\caption{Cross-species hippocampus anatomy}
  	\label{fig:hippocampusAnatomy}
\end{figure}

\subparagraph{Hippocampal Functions} In London taxi drivers were observed an increased brain activity associated with spatial navigation in the \textbf{right hippocampus} and left tail of the caudate. In rats the effect of hippocampal lesions on reference learning and memory was tested using the Morris water maze experiment. As bigger is the lesion on dorsal hippocampall, as bigger the deficit in the acquisition of spatial reference. Not so big deficit if the lesion were in the ventral hippocampal.

Experiment: The position of a submerged platform is constant from trial to trial at a given test day as well as form test day to test day. Animals are repeatedly placed into the tank with varying starting positions; with the help of spatial distal cues as reference points, they are required to find the invisible platform. Following completion of the acquisition phase, the platform is removed from the tank. The animals are once again placed in the tank; the critical measure here is whether the animals would “remember” the position of the platform and therefore would spent more time in quadrant where the platform was positioned before.

\paragraph{Amygdala}

\subparagraph{Amygdalar Anatomy} Primary amygdalar nuclei and basic circuit connections and
function are conserved across species. An enlarged image of the basolateral complex of the
amygdala (BLA) and central nucleus of the amygdala (CeA) or analogues are shown next to a coronalsection from the brains of a lizard, rat, cat, monkey, and human, in Figure \ref{fig:amygdalaAnatomy}.

\begin{figure}
	\centering
  	\includegraphics[width=\linewidth]{imgs/amygdalar-anatomy.png}
	\caption{Cross-species amygdalar anatomy}
  	\label{fig:amygdalaAnatomy}
\end{figure}

\subparagraph{Amygdalar Functions} In post-traumatic stress disorders (PTST), the amygdala is hyperactive in response to negative emotional stimulli vs. neutral and positive stimulli. In rodents the investigation of amygdalar function is tested using the \textbf{classical (pavlovian) fear conditioning}. In rats with amygdala lesions, the response to the non-threatening doesn't happen anymore.

Experiment: present a non-threatening stimulus (like a sound) with a noxius stimulus (like a midle shock) until the animal shows a fear response not just to the shock but also to the sound alone.

\paragraph{Basal ganglia}

\subsection{Exercises}

\subsubsection{Coronal section - I}
\begin{figure}[H]
	\centering
  	\includegraphics[width=\linewidth]{imgs/coronal-section-I.png}
	\caption{Coronal section I}
  	\label{fig:coronalSectionI}
\end{figure}

\subsubsection{Coronal section - II}
\begin{figure}[H]
	\centering
  	\includegraphics[width=\linewidth]{imgs/coronal-section-II.png}
	\caption{Coronal section II}
  	\label{fig:coronalSectionII}
\end{figure}

\subsubsection{Coronal section - III}
\begin{figure}[H]
	\centering
  	\includegraphics[width=\linewidth]{imgs/coronal-section-III.png}
	\caption{Coronal section III}
  	\label{fig:coronalSectionIII}
\end{figure}

\subsubsection{Coronal section - IV}
\begin{figure}[H]
	\centering
  	\includegraphics[width=\linewidth]{imgs/coronal-section-IV.png}
	\caption{Coronal section IV}
  	\label{fig:coronalSectionIV}
\end{figure}

\subsubsection{Horizontal section}
\begin{figure}[H]
	\centering
  	\includegraphics[width=\linewidth]{imgs/horizontal-section.png}
	\caption{Horizontal section}
  	\label{fig:horizontalSection}
\end{figure}

\section{Molecular \& Cellular Neuroscience}
\subsection{Building a central nervous system}
Human brain: 86 billions neurons and about equal number of glia cells.

\subsubsection{Neural Induction and Pattern Formation}

\paragraph{Embryonic Origins of the Nervous System}

\begin{figure}
	\centering
  	\includegraphics[width=\linewidth]{imgs/embryonic_origins_nervous_system.png}
	\caption{Embryonic origin of the neural system}
  	\label{fig:embrionicOriginNS}
\end{figure}

Figure \ref{fig:embrionicOriginNS}: the \textit{ectoderm} (blue/red in image) covers the outside of the embryon during \textit{gastrulation}.\footnote{Gastrulation is a phase early in the embryonic development of most animals, during which the single-layered \textit{blastula} is reorganized into a trilaminar ("three-layered") structure known as the \textit{gastrula}. These three germ layers are known as the ectoderm, mesoderm, and endoderm.} Ectodermal cells give rise to different derivatives depending on position along the dorsoventral (DV) axis of the embryo. The dorsal-most ectoderm (red) thickens to form the \textbf{neural plate}, a structure shaped like a tennis racquet with the head lying anteriorly. During a complex morphogenetic process called \textbf{neurulation}, the flat neural plate rolls up into a tube that sinks into the interior of the embryo and becomes overlain by epidermal ectoderm. This neural tube is the anlage of the central nervous system (CNS). As the neural plate folds and closes, neural crest cells detach from its lateral margins and migrate away, later condensing to form the major part of the peripheral nervous system (PNS).

\paragraph{The famous Spemann Organizer}

\begin{figure}
	\centering
  	\includegraphics[width=0.5\linewidth]{imgs/spemann_organizer.png}
	\caption{Spemann Organizer Experiment}
  	\label{fig:spemannOrganizer}
\end{figure}

Figure \ref{fig:spemannOrganizer}: Tissue around the DBL\footnote{dorsal blastopore lip, where mesodermal cells start to involute during gastrulation} was removed from one embryo
(black) and placed into the ventral side of another (white). The transplanted DBL, if large enough, will cause a complete second dorsal axis to form on the host embryo, resulting in twinning. Cross section through the tadpoles shows that the second dorsal axis contains a complete nervous system. By using differently pigmented embryos, one can show that the majority of the nervous system in this new dorsal axis is not derived from the transplanted tissue, but rather from host tissue, fated to give rise to ventral tissues in the absence of a graft.

\paragraph{The default model for Neural Induction}
An important aspect of the default model for neural induction is that ectoderm will inherently (by default) form neural tissue unless it is exposed to antineuralizing signals.

\begin{figure}
	\centering
  	\includegraphics[width=\linewidth]{imgs/neural_induction.png}
	\caption{Default model for neural induction}
  	\label{fig:neuralInduction}
\end{figure}

Figure \ref{fig:neuralInduction}: Experiments in Xenopus embryos that led to the default model: culture of animal cap explant results in epidermis differentiation; dissociation for several hours followed by reaggregation of animal cap tissue results in neural induction; the presence of BMPs during dissociation prevents neural induction and promotes epidermis formation; the expression of a dominant-negative Activin receptor results in neural induction even without dissociation. BMPs induce epidermal fate and inhibit neural induction via Smad1 activity; BMP inhibitors act as neural inducers by blocking BMPs; FGFs act as neural inducers by counteracting Smad1 and via BMP-independent mechanisms; Wnt/$\beta$-catenin signaling predisposes ectoderm for neural induction by both preventing the transcription of Bmp genes and inducing the expression of BMP inhibitors.

\paragraph{Early Neural Patterning}
The neural plate is parcellated into subdivisions along the anterior-posterior (AP) and dorsalventral (DV) axes. This subdivision give rise to a multitude of different cell types in a spatially organized manner; these then become wired up in a stereotyped fashion and subserve distinct tasks in functional neural networks.

\subsubsection{Growth Cones and Axon Pathfinding}
\subsubsection{Cellular Determination}
\subsubsection{Neurogenesis and Migration}

\subsection{Excitability}
\subsection{Glia and more}
\textbf{Glial cells}: non-neuronal cells that maintain homeostasis, form myelin and provide support and protection for neurons in CNS and PNS. They are divided in \textit{macroglia} and \textit{microglia} cells.
\begin{itemize}
\item microglia destroy infectious agents and work as imune defense system of the brain through phagocytosis.
\item macroglia divided in two groups: astrocytes and oligodendrocytes.
\end{itemize} 

\subsubsection{Astrocytes}

Five functions:
\begin{itemize}
\item Energy metabolismo: provide O2 to neurons from blood
\item Waste recycling: remove CO2 from neurons to blood
\item Neurotransmission: convert glutamate in glutamine to neurons
\item Biosynthesis: produce glutamine 
\item blood flow regulation: regulates contraction and dilatation of blood vessels
\end{itemize}
\subsubsection{Oligodendrocytes}
In CNS they are called olygodendrocytes, in PNS Schwann cells. Responsible for myelin formation.

Myelin is responsible for:
\begin{itemize}
\item critical in increase action potential conduction
\item provide metabolic support
\item affect signal processing and long distance communication by modulating the degree of myelination
\end{itemize}

\textbf{Plasticity}: the myelination changes throughout adult life based on experience and neuronal activity.
\subsection{Synapses}

Most floe information is chemical: a single pre action potential can generate a large postsynaptic potential. 

\subsubsection{Neurotransmitters}
Interneuronal communication is chemical in nature but neurons also use other process to communicate (as electrical, ephatic iteration etc). \textbf{Neurotransmitters} are substances thar are released from neurons, act on receptors of postsynaptic cells and produce a functional change in the target cell.

\paragraph{Mainly communication types:} electral and chemical. Each type of communication has specific channels. The chemical communication occurs by syanpses and the electrical communication occurs by gap junctions.
\textbf{synapse}: gap between two neurons (synapse cleft). The transmission is not all or none, but it is graded.
\textbf{gap junction}: two neurons are connected by gap junctions. When one fires, the other fires simultaneously and with the same intensity.

There are diverses types of neurotransmitters. However, there is a classical structure of a neurotransmitter:
\begin{itemize}
\item neuron must produce and release the substance
\item substance must be release from nerve terminals
\item substance should reproduce at postsynaptic the event seen on the presynaptic
\item should have mechanisms to terminate the action of the substance
\end{itemize}

This rules are related with the five steps of the chemical neurotransmission (communication):
\begin{enumerate}
\item Synthesis of neurotransmitter in the presynaptic neuron
\item Storage of the neurotransmitter (or its precursor) in the presynaptic nerve terminal
\item Release of neurotransmitter into the synaptic cleft
\item Binding and recognition of neurotransmitters by target receptors
\item Termination of the action of the released neurotransmitter
\end{enumerate}

There are two type of receptors: ionotropic and metabotropic.
\begin{itemize}
\item ionotropic: are made from proteins to form an ion channel. the transmitter binds with the receptor and opens the channels for the ions going through. Ionotropic receptor have rapid changes with short duration.
\item metabotropic: are made of a single peptide and the transmitter binds with the receptor and then, the g-protein is activated. The activation of g-protein will open the ion channel. Usualyy is necessary a second messeger to do this. Metabotropic receptor have slower response but with long duration.
\end{itemize}

There is two main types of release of neurotransmitters: kiss and run and endocytosis.
\begin{itemize}
\item kiss and run: a fusionpore opens to allow transmitter release and then closes rapdly to reform the vesicle, this way, the vesicle is available immediatly to reuse.
\item endocytosis: vesicle fuses with the membrane completely to release transmitters. The reformation of the vesicle requeries that appropriate proteins be reassembled
\subitem the vesicle can return directly to the release pool
\subitem the vesicle first fuses with endosome and the a new vesicle is generated
\end{itemize}

There are two ways to finalize the action of a neurotransmitter: active and passive.
\begin{itemize}
\item active: glia cells cleans the neurotransmitters or by enzymatic degradation
\item passive: diffusion of the transmitter in the enviroment.
\end{itemize}

\paragraph{Why so many types of neurotransmitters?} There is many terminal synapses onto a single neuron. This way to distinguish different information they need different chemical codes.


\section{Systems Neuroscience}
\subsection{Somatosensory and Motor Systems}

Three groups of receptors: \textbf{mechanoreceptor}, \textbf{nociceptor} and \textbf{thermoceptors}.
The nociceptors and thermoceptors are free nerve endings: unmyelinated terminal on derme/epiderme.
The myelinated type (mechanoreceptor) is divided in four major types with two subcategories. The subcategories are slow and fast adaptation. The \textbf{slow adaptation} means the neuron fires while the stimulus occurs, the \textbf{rapid adaptation} fires in the beggining then adapt and stops.

\begin{itemize}
\item meismer's corpuscles: rapid adaptation, closest to epiderm
\item pacinian corpuscles: rapiid adaptation, high vibrations
\item merkel's disks: slow adaptation, located in epiderm, identification of shapes, edges and textures
\item ruffini's corpuscles: slow adaptation, deep in the skin (ligaments, tendons), sensitive to stretching.
\end{itemize}

The CNS plays an active role in determine perception, one interesting case is the phantom limb. The phantom limb occurs when people lose their limbs and still have the sensation of pain or other stimulus from the missing limb.

\subsubsection{Somatosensory Pathway}
Sensory pathway:
receptors receives the stimulus and transduce it to pass the information to the first order neurons (neurons that have cell body on dorsal root ganglion or in the trigeminal ganglion). After the axon of the first order neurons enter the spinal cord they branch into ascending and descending pathways. The major branch ascend ipsilaterally throught the dorsal column to medulla.


\subsubsection{Motor pathway}
Definitions:
\begin{itemize}
\item motor neurons: body along the spinal cord and brainstem. Send axons to one muscle and innervate some muscle fibers.
\item motor unit: motor neuron + muscle fibers.
\end{itemize}

An individual muscle is controlled by a pool of motor units. This pool contains varying portion of motor unit types. The pool of motor neurons forms an enlongated column that extends for two or three segments of the spinal cord.

Three types of muscle fibers formed by contraction and metabolic properties:
\begin{itemize}
\item Type I: slow twitch and aerobic (oxidative) metabolization
\item Type IIA: fast twitch and aerobic and anaerobic metabolization
\item Type IIB: fast twitch and anaerobic (glucolysis) metabolization
\end{itemize}

Each motor neuron must link with fibers of the same type. This way, also the motor neurons are divided in three types:
\begin{itemize}
\item Type S: slow twitch, lower production of force and resistant to fatigue. Linked with type I fibers. They small cell bodies, responsible for postural and tonic movements.
\item Type FR: fast twitch, relative high amount of force produced and relative resistance to fatigue. Linked with Type IIA fibers. They have large cell bodies and faster firing rates, responsible for fast and powerfull movements.
\item Type FF: fast twitch, higher amount of force produced, fatigue quickly. Linked with Type IIB fibers. They have large cell body and fast fireing rates, responsible for brief burst of muscle strength.
\end{itemize}



Motor pattern reflex:
\begin{itemize}
\item Withdraw of a part of the body to avoid pain or tissue damage
\item Coughing or sneezing to remove an irritant from nasal or tracheal mucosa
\item Swallowing reflex to propel the food down the esophagus to stomach
\end{itemize}

Diencephalon and subcortical areas of telencephalon controls the goal-direct behavior (hypothalamus and basal ganglia). The hypothalamus controls autonomic functions (temperature regulation, intake fluid or food). The basal ganglia has critical importance for normal initiation of motor behavior (thalamus: input, pallidum: output). The frontal lobe contains minor neurons that give us capacity to imitate movements.

Corticospinal neurons: large pyramidal cells in the motor cortex that send their axons to the contralateral side of the spinal cord. They are able to activate they target motor neurons directly.

Cortical influence on movements by direct projections to the input of basal ganglia or to the spinal cord (corticospinal projections). Responsible for conscious decision about initiate and maintain movement and to adaptation using visually information.

Two main system in motor:
\begin{itemize}
\item Medial system - axons descend through brainstem and spinal cord close to the midline: postural control.
\item Lateral system - axons descend in the lateral column of spinal cord: fine control of voluntary movements.
\end{itemize}

Convergence and divergente on primary motor cortex:
\begin{itemize}
\item convergence: any given muscle is controlled by a large territory in M1.
\item divergence: single M1 neurons have output connections that diverge to innervate the spinal motor neuron pools of multiple muscles.
\end{itemize}


\paragraph{Basal Ganglia Loops}
There is a parallel organization on basal ganglia: cortex $\rightarrow$ striatum $\rightarrow$ pallidum $\rightarrow$ thalamus $\rightarrow$ cortex

Motor loop: primary motor cortex $\rightarrow$ putamen $\rightarrow$ lateral globus pallidus $\rightarrow$ ventral lateral and ventral lateral anterior nuclei.

Oculomotor loop: posterior parietal pfc $\rightarrow$ caudate $\rightarrow$ globus pallidus internal $\rightarrow$ mediodorsal and ventral anterior nuclei.

Prefrontal loop: dorsolateral pfc $\rightarrow$ anterior caudate $\rightarrow$ globus pallidus internal $\rightarrow$ mediodorsal and ventral anterior nuclei.

limbic loop: amygdala, hippocampus $\rightarrow$ ventral striatum $\rightarrow$ ventral pallidum $\rightarrow$ mediodorsal nucleus

\subsection{Visual System}
\subsection{Auditory \& Vestibular System}
\subsection{Circuits underlying Emotion}
\subsection{Learning in artificial and biological neural networks}

\section{Answers}
\subsection{Human \& Comparative neuroanatomy}

\subsubsection{Coronal section - I}
\begin{figure}[H]
	\centering
  	\includegraphics[width=\linewidth]{imgs/coronal-section-I-answer.png}
	\caption{Coronal section I}
  	\label{fig:coronalSectionI-answer}
\end{figure}

\subsubsection{Coronal section - II}
\begin{figure}[H]
	\centering
  	\includegraphics[width=\linewidth]{imgs/coronal-section-II-answer.png}
	\caption{Coronal section II}
  	\label{fig:coronalSectionII-answer}
\end{figure}

\subsubsection{Coronal section - III}
\begin{figure}[H]
	\centering
  	\includegraphics[width=\linewidth]{imgs/coronal-section-III-answer.png}
	\caption{Coronal section III}
  	\label{fig:coronalSectionIII-answer}
\end{figure}

\subsubsection{Coronal section - IV}
\begin{figure}[H]
	\centering
  	\includegraphics[width=\linewidth]{imgs/coronal-section-IV-answer.png}
	\caption{Coronal section IV}
  	\label{fig:coronalSectionIV-answer}
\end{figure}

\subsubsection{Horizontal section}
\begin{figure}[H]
	\centering
  	\includegraphics[width=\linewidth]{imgs/horizontal-section-answer.png}
	\caption{Horizontal section}
  	\label{fig:horizontalSection-answer}
\end{figure}

\newpage
\section{Previous Exams}
Note this answers were provided by students and were not verified by a teacher. Use them at your own risk.

\subsection{2016}
\paragraph{Q1. Anatomy}
\subparagraph{Label Coronal section III as in Fig:\ref{fig:coronalSectionIII}.}
\subparagraph{Another image of the brain to labeling as the first one of Figure \ref{fig:boundaries}.}
\subparagraph{Define the parts of the prefrontalcortex in human ans in rats, also describe how they impact in working memory.}
\paragraph{Q2. Neurogenesis}
\subparagraph{Define gradient}
\subparagraph{Choose an example where gradient are important in neurogenesis. Define how this gradient is read out, ... }
\paragraph{Q3. Synapses}
\subparagraph{Define at least five experiments about the importance of Ca2+ for neurotransmission?}
\subparagraph{What is the flow of the Ca2+ in the synapse membrane? How the Ca2+ is transported?}
\subparagraph{more question involving Ca2+}
\paragraph{Q4. Excitability}
\subparagraph{Consider the voltage-clamp experiment of a single channel. The following graph was obtained: image of a linear current-voltage graph. Why the current change when the voltage change?}
\subparagraph{What is E\_{rev}?}
\subparagraph{What is single channel conductance (g)?}
\subparagraph{Why some ion channels allows an specific type of ion going through, for instance, why potassium channel allow only potassium ions to going through?}
\subparagraph{What is channel inactivity?}
\subparagraph{You analise a ALS patient with gPI (?). What you observe?}
\paragraph{Q5. Auditory and Vestibular pathway}
\subparagraph{Define the cortical and the subcortical areas envolved in the ascendy auditory pathway, namely the parts where the information cross the midline}
\subparagraph{How the horizontal spatial localization is calculated in the SOC?}
\subparagraph{What are the five organs of the vestibular pathway? How they are involved in rotation and linear moviments of the head?}
\paragraph{Q6. Biological and artificial learning}
\subparagraph{Consider the experiment: an sound is played before an air puff in the subject eye. In respond to the air puff, the subject blinks. After a while, when the sound is played, the subject blinks. In this example, define: CS, US, CR, UR.}
\subparagraph{Rescola-Wagner: what is the aim of RW rule? What is the formula? In the previous example, what are the weights before and after learning? What are the weights if the air puff occurs only in 50\% of the times that the sound is played?}
\subparagraph{What is the Hebb's postulate? Is the RW rule related with the Hebb's postulate? If yes, explain how. }

\subsection{2013}
\paragraph{Q1. 5 methods (advantages + limitations) to label CNS neurons in rodent.}
\paragraph{Q2. Label 15 parts on a coronal slice (slice in which you see pons)}
\subparagraph{Describe how your project is related to some brain parts OR talk about the midbrain and its functional parts (+labelling a drawing of it).}
\paragraph{Q3. Chemical + electrical synapses}
\paragraph{Q4. A man with the “man who lost his body” documentary problem, how to test for his condition.}
\paragraph{Q5. Neurogenesis areas, labelling techniques, positive regulators and diseases associated with problems in neurogenesis.}
\paragraph{Q6. Bird auditory neuron behaviors, setup of the experiment with the bird and why is it important that the bird does not hear any external sounds.}

\subsection{2012}
\paragraph{Q1. Neuroanatomy}
\paragraph{Q2. Somatosensory}
\paragraph{Q3. Vision system}
\paragraph{Q4. Neural computation}
\paragraph{Q5. Brain development: neurogenesis}
\paragraph{Q6. Ion channel or synaptic transmission}

\subsection{2011}
\paragraph{Q1. Discuss the functions and structures of the hypothalamus as discussed in the lecture material.}
\subparagraph{Label 18 structures in 2 different coronal slices} see exercises.
\paragraph{Q2. Describe how DRG (dorsal root ganglion) sensory neurons development in comparison to motor neurons. How are cell boundaries formed in general and among the specific motor/sensory nerves}
\paragraph{Q3. Axon Guidance: what were sperry's findings that support the chemoaffinity hypothesis. What molecules are involved in this and how do they function.}
\paragraph{Q4. Describe from how sound is encoded neurally (from entering the ear to being perceived as sound in brain - complete pathway)}
\paragraph{Q5. Draw a flowchart for a typical neuroproteomics experiment}
\paragraph{Q6. Fill in the blank and multiple choice questions from Tobi's lecture: Who invented the term Neuro Engeneering? What is CMOS? Power consumption of brain. Synchronous logic is ubiquitous slide know physiologists friend photodiodes - how they are similar to retina CARVER MEAD}


\subsection{2010}
\paragraph{Q1. Auditory pathway}
\paragraph{Q2. Development of CNS and PNS}
\paragraph{Q3. Boundary building (one slide, different cell type)} ????
\paragraph{Q4. Pathfinding (Chemoaffinity, give 2 examples)}
\paragraph{Q5. Anatomy (hypothalamus, position and function)}
\paragraph{Q6. Neuromorphic engineering}

\subsection{2009}
\paragraph{Q1. Neuroanatomy: which of the 12 cranial nerves origin and/or end in the brainstem? What are their respective sensory, motor and /or vegetative functions ?(please describe in detail) Which nuclei of the cranial nerves are located in the mesencephalon?}
\paragraph{Q2. Auditory system: Describe differences between "conductive hearing loss" and "sensorineural hearing loss". Describe the classical test which is often used to determine between both forms of hearing loss. Describe biological causes and current treatments aids for such hearing impairments.}

\paragraph{Q3. Proteomics in neuroscience:}
\subparagraph{a. explain the term "proteome"}
\subparagraph{b. what are the benefits of measuring the proteome in addition to the genome?}
\subparagraph{c. Describe what a mass spectrometry is doing in principle.}
\subparagraph{d. How would you quantify proteins in a proteomic experiment? Please name and describe at least 2 proteomics technologies}
\subparagraph{e. Why is the proteome more complex compared to genome? Name and describe 3 reasons.}
\paragraph{Q4. Ion channels: What are the principal functions of dendrites, axon and nerves endings in the transcription of signals through the nervous system? Which types of ion channels are critical for the function of each of these 3 structures? Provide specific examples.}
\paragraph{Q5. Neural network: Explain the temporal and spatial network definition. Give an example for each network definition and describe how you can detect these networks in the brain.}
\paragraph{Q6. Neuromorphic engineering: Considering organizing principles used in biological retina explain (...)}

\subsection{2008?}
\paragraph{Q1. Describe the diencephalon and its major components according to the text "the brain in a nutshell"}
\paragraph{Q2. Compare structure and development of the cerebellum and the cortex}
\paragraph{Q3. What evidence did Sperry find that supports his chemoaffinnity hypothesis? Have Sperrys proposed "recognition molecules" been found? If yes name one example and describe what properties of this molecyles support its role as a recognition molecule}
\paragraph{Q4. Describe the structure of a voltage potassium channel. Explain the mechanisms that make the channel selective for only potassium ions.}
\paragraph{Q5. Describe three functional properties of neurons in v1 that are absent in the LGN. For each property describe in detail an experiment that illustrates it including the type of stimulus and the observed neural responses. Finally, choose one of these three properties and explain as presisely as possible how it can emerg at the cortical level.}
\paragraph{Q6. Which dynamic processes occur in single neuron and the local neural circuit during signal flow through a neural network? Name critical structural and functional aspects and discuss how they can be measured experimentally}

\subsection{2007}

\paragraph{Q1. Label each part of the brain, two coronal section, 18 areas.} see exercises.
\subparagraph{Describe the lobes of cortex, according to the handout.} 
\paragraph{Q2. Compare the cell migration to form the cortex and the migration in the peripheral neural system forming…} answer
\paragraph{Q3. About neurotrophic factor. What’s the experiment led to the finding of neurotrophic factor? Compare trophic and tropic factor.} answer	1
\paragraph{Q4. What’s the difference between ionotropic and metabotropic receptors?} Ionotropic receptors are made from proteins combined to form an ion channel. The neurotransmitter binds the receptor and open the channel for the ions going through. Ionotropic receptor have rapid changes with short duration. Metabotropic receptors are made of a single peptide and the neurotransmitter binds with a g-protein instead of the ion channel directly. It is need a second messenger to open the ion channel. Metabotropic receptors have slower response but with long duration.
\paragraph{Q5. Serotonin}
\paragraph{Q6. Insect eye}

\subsection{2006}
\paragraph{Q1. Label each part of the brain.} see exercises.
\subparagraph{Describe the components of midbrain according to the description in the shells of the brain}
\paragraph{Q2. Why the ion channel is selective to a ion with particular polarity, e.g. why the positive channel allows cations going through? List different ion channels in different part of the neuron: (1) axon, (2) axon terminal and (3) postsynaptic membrane}
\paragraph{Q3. How does cells differentiate in neural tube? How does cells differentiate in peripheral nervous system?}
\paragraph{Q4. What is the structure and function of myelinated nerve in PNS?}
\paragraph{Q5. Describe the structure and function of dopaminergic systems, the relation between the systems and psychology and neuropharmacology}
\paragraph{Q6. Visual system: describe the two columns in V1, and give out the definitions of them. What is the relation between this two columns and also between them and other part of visual system. Give out one experiment for testing of each column.}


\subsection{2005}
\paragraph{Q1. Draw the connectivity between motor cortex, thalamus, basal ganglia and cerebellum for motor control and show which connections are excitatory or inhibitory}
\subparagraph{Coronal view to fill with 16 names} see exercises.

\paragraph{Q2. Model organisms for understanding human brain development and function. Give three general advantages of these models. Compare in a table with advantages and disadvantages the models: Drosophila, C. Elegans, Zebrafish, Mouse. Give an example for each of how it helped understand the brain}

\paragraph{Q3. Two main route for cell death, detail differences}

\paragraph{Q4. Myelin: structure and function}

\paragraph{Q5. What are the two main modes of electrophysiological communication between neurons? Describe structure.}

\paragraph{Q6. Vision:}
\subparagraph{A. cites six roles for vision in insects}
\subparagraph{B. 1. what structural/functional differences between insect and human eyes 2. in what experiments are insects eyes worse or better than human eyes? 3. five share important features of insect and human eyes}
\subparagraph{C. 1. what is flow field? 2. draw the flow field perceived by a fly flying straight in a long corridor 3. what is ... field? ( i forgot the name oops) 4. draw pure rotation force field and its matched field filter}


\subsection{2004}
Some questions (especially number 6) are about subject not taught anymore during the first ZNZ introduction semester, so don't worry about them.

\paragraph{Q1. Describe the major formations involving the hippocampus in the associative cortex.}
\subparagraph{Coronal slices, 16 areas to be labelled} see exercises.
\paragraph{Q2. Structure and role of myelinating cells in the adult nervous system.}
\paragraph{Q3. Name some crucial functions of Neurotrophic factors.}
\paragraph{Q4. How is information transported in the nervous system? Explain features and function.}
\paragraph{Q5. In verterbrates the vision system has some special wiring pattern. What's special about it (as in, how is it different to olfaction)? Explain biological/physiological means in the development of vision.}
\paragraph{Q6. Imagine year 2020. Human genomics has advanced to the point where you not only can choose the gender and hair color of your child, but also apply specific changes to the visual system. Name 6 changes to the human visual system you would apply to your kid. Explain why you chose them and what physiological implications they would have.} See Visual System, question \ref{question:year2020}.

\newpage
\subsection{All Question - topics}

Here you can find a pull with question of previous exams and questions from self-study (PQ\footnote{PQ stands for personal question}), that is, questions marked as PQ are not question listed in previous exams.

%%%%%%%%%%%%%%%%%%%%%%%%%%%%%%%%%%%%%%%%%%%%%%%%%%%%%%%%%%%%%%%%%%%%%%%%%%%%%%%%%%%%%%%%%%%%%%%%%%%
\subsubsection{Cytology}
\begin{enumerate}
\item \paragraph{What is the structure and function of a myelinated peripheral nerve?}
Structure: ???

The myelin in the peripherical nervous system is generated from Schwann cells. Myelin is a fatty substance (composed about 75\% of lipids, 20\% of proteins and 5\% of carbohydrates) that surrounds the axon of some nerve cells, in the PNS, the myelin is formed by many layers of schawnn cells membrane. The mainly function of a myelinated nerve is increase action potential conduction, that is, the message is delivered faster when compared with an unmyelinated nerve. The myelin in peripheral nerve also provides a track along with a severed nerve can regrowth. This regeneration do not happens in unmyelinated nerver or in myelinated nerves in CNS.

\item \paragraph{Myelin : structure and function}
Structure?

Myelin is a fatty substance (about 75\% of lipids, 20\% of proteins and 5\% of carbohydrates) that surround the axon of some nerve cells.
The maily function of the myelin is increase the action potencial conductancy, that is, to make the signal be propagated faster. In CNS the myelin is generated by olygodendrocytes and in the PNS by Schawnn cells.
In the PNS the myelin helps severed nerves to regrow, and unmyelinated and myelinated from CNS can't regenerate.

\item \paragraph{Structure and role of myelinating cells in the adult nervous system.}
Structure?
Adult nervous system... this differ anything from previous answers?

\item \paragraph{Serotonin}
Serotonin (or 5-HT) is a neurotransmitter. It is popularly thought to be a contributor to feelings of well-being and happiness.
The serotonin receptors are metabotropic receptors (that is, they bind with a receptor that will activate a g-protein) but 5-HT3 is a ionotropic receptor. The alteration on serotonin levels can cause some disorders: the decay of serotonin can cause depression, anxiety and even social phobia. However, higher levels of serotonin can cause serotonin syndrome (with confusion, loss of balance, fever etc).

\item \paragraph{Two main route for cell death, detail differences}
There are two main ways to programmed cell death: autonomous specification and conditional specification.
\begin{itemize}
\item autonomous specification: involves the segregation of cytoplasmic molecules by asymetric division. This way, cell death is programmed into the lineages that generate somatic cells.
\item conditional specification: involves external signals. One example is when the axons do not find the cues to grow or do not make connection with their target. The absence of this connection generates a programmed cell death.
\end{itemize}

\item \paragraph{Proteomics in neuroscience:}
\subparagraph{a. explain the term "proteome"}
\subparagraph{b. what are the benefits of measuring the proteome in addition to the genome?}
\subparagraph{c. Describe what a mass spectrometry is doing in principle.}
\subparagraph{d. How would you quantify proteins in a proteomic experiment? Please name and describe at least 2 proteomics technologies}
\subparagraph{e. Why is the proteome more complex compared to genome? Name and describe 3 reasons.}?
\end{enumerate}


%%%%%%%%%%%%%%%%%%%%%%%%%%%%%%%%%%%%%%%%%%%%%%%%%%%%%%%%%%%%%%%%%%%%%%%%%%%%%%%%%%%%%%%%%%%%%%%%%%%
\subsubsection{Anatomy}
\begin{enumerate}
\item \paragraph{Label 2 coronal slices, 16/18 areas to be labelled} See exercises: Figures \ref{fig:coronalSectionII-answer} and \ref{fig:horizontalSection-answer}.

\item \paragraph{Label 15 parts on a coronal slice (slice in which you see pons)} See exercises: Figure \ref{fig:coronalSectionIV-answer}.

\item \paragraph{Describe the major formations involving the hippocampus in the associative cortex.} 
The hippocampus is located at the medial temporal lobe and it is a structure that resembles a seahorse. The hippocampus is divided in three major parts: the dentate gyrus, the subiculum and the cornu ammonis. The hippocampus is associated with learning and memory. The consolidation of short term memories in long term memories involves the hippocampus.
The associative cortex is the cortex part outside the primary areas. It is essential for mental functions that are more complex that basic detection of sensory stimulation. Each sensory (primary) system has its own associative cortex areas.
The hippocampus receives input from all the association areas and sends signals back to them as well as other, this way, the hippocampus creates new associations (or learning). The hippocampus associates the current features of the perceived object with other older memories related with the same object. This creates a rich multi modal memory. For instance, the memory of a bird is associated with its sound.

\item \paragraph{Mesencephalon: components \& nuclei (brain in a nutshell)}\label{question:mesencephalon}
The mesencephalon, also know as midbrain is divided in two major parts: the tectum (roof) and the tegmentum (floor). This two parts are separated for the cerebral aqueduct (a tyne canal that connects the third and fourth ventricule).
The tectum is made of the culliculli, that are divided in superior culliculus and inferior culliculus.
\begin{itemize}
\item the superior culliculus is responsible for visual and occulomotor reflex 
\item the inferior culliculus is responsible for sound reflex (relay the audictory tract).
\end{itemize}
The tegmentum is divided in five structures: reticular formation, red nucleus, substancia nigra, cerebral peduncles and ventral tegmental area.
\begin{itemize}
\item reticular formation: is involved in automatic processing of incoming sensations and outcoming motor comands, and helps to maintan the consciouness
\item red nucleus: is involved in motor coordination (involuntary control of muscle tone and limb posture)
\item substancia nigra: is the only part of the brain that contains melanin. Also contains dopaminergic neurons. It is involved in regulation of the basal nuclei activities. Is divided in two parts:
\subitem substancia nigra pars compacta: formed by dopaminergic neurons. The degeneration of these neurons cause Parkinson's disease.
\subitem substancia nigra pars reticulata: contains most of neurons as inhibitory, that release GABA. 
\item cerebral peduncles: connect primary motor cortex with motor neurons in brain and spinal cord. Contains the large ascending (sensory) and descending (motor) nerve tracts.
\item ventral tegmental area: part of the limbic system, involved in emotional reinforcement, projects to nucleus accumbens and amygdala.
\end{itemize}

\item \paragraph{Motor activity structures and fibers}
Motor neuron: neurons with the nucleus along the spinal cord and brainstem, they send their axons to one muscle and innervate some muscle fibers. All the muscle fibers must be of the same type.
Muscle fibers are classified accordlying with contract properties (slow or fast) and metabolic properties (aerobic or anaerobic). There are three types of muscle fibers:
\begin{itemize}
\item Type I - fibers with slow twitch and good oxidative (aerobic) metabolism
\item Type IIA - fibers with fast twitch and oxidative and both aerobic and anaerobic metabolism: with good resistance to fatigue)
\item Type IIB - fibers with fast twitch and anaerobic (glycolysis) metabolism.
\end{itemize}

An individual muscle is controlled by a pool of motor neurons (varying in portion of motor units).
A motor unit is formed by one motor neuron and all related muscle fibers, as the muscle fibers, motor units also are classified in three types:
\begin{itemize}
\item Type S - slow twitch, small amount of force produced and high resistance to fatigue
\item Type FR - fast twitch, moderate force produced and relatve resistance to fatigue
\item Type FF - fast twitch, highers force produced and fatique quickly
\end{itemize}

Each type of motor unit is related with a muscle fiber type, so, Type S motor unit is associated with Type I muscle fibers, the Type FR with Type IIA and Type FF with Type IIB.

The pool of motor neurons forms an elongated column that extends over two or three segments, and they can be of three types: $\alpha$, $\beta$ and $\gamma$. $\alpha$ motor neuron innervates skeletal muscle fibers, $\gamma$ motor neuron innervates muscle spindles and $\beta$ motor neuron innervate both types.

\item \paragraph{Output structures and structures modulating output} ???

\item \paragraph{Draw the connectivity between motor cortex, thalamus, basal ganglia and cerebellum for motor control and show which connections are excitatory or inhibitory}

\item \paragraph{Describe the diencephalon and ist major components according to the text "the brain in a nutshell"}
The diencephalon is divided in four major parts:
\begin{itemize}
\item thalamus: is the gatekeeper of the brain, that is, is important for the transfer of information from the periphery to sensory processing regions in the telencephalon. It determines wheter sensory information reaches conscious awareness in the neocortex. It participates of the integration of motor information from the cerebellum and basal ganglia to cerebral areas concerned with movement.
\subitem TODO: explain thalamus 
\item hypothalamus: regulates behavious essential for homeostasis and reproduction: growth, eating, drinking and maternal behaviour by regulating hormonal secretions from pituitary gland. It is an important control center for the autonomic nervous and stress-response system.
\item epithalamus: epithelial roof of the third ventricle. It is responsible for regulation of day-night cycles and information processing related to olfaction.
\item subthalamus: it is continuation of the tegmentum, functionally part of the basal ganglia.
\end{itemize}

\item \paragraph{Discuss the functions and structures of the hypothalamus as discussed in the lecture material.}\label{question:hypothalamus}
The hypothalamus is located under the thalamus and above the pituitary gland and the brain stem. It contains important collection of nuclei and is about of the size of an almond and it is part of the limbic system. The hypothalamus is mainly responsible for homeostasis of the body and for production of many essential hormones. The hypothalamus produces releasing and inhibitory hormones that influence anterior pituitary hormone secretion. It produces oxytocin and antidiuretic hormones that are stored in the posterior pituitary. Besides that, the hypothalamus oversees the autonomic neural system, helping to stimulate the adrenal medulla via sympathetic innervation.

\item \paragraph{Anatomy (hypothalamus, position and function)}
see question \ref{question:hypothalamus}

\item \paragraph{Which of the 12 cranial nerves origin and/or end in the brainstem? What are their respective sensory, motor and /or vegetative\protect\footnote{out-of-date term for the autonomic nervous system} functions? (please describe in detail). Which nuclei of the cranial nerves are located in the mesencephalon?}

The Autonomic Nervous System (ANS) is comprised of both sensory and motor components, which involuntarily tend to homeostasis. Autonomic sensory neurons transmit information to the CNS, via autonomic sensory receptors which are primarily situated, in the visceral organs (smooth muscle organs in the thorax, abdomen, and pelvis). The Sesory Neurvous System (SNS) consists of somatic sensory neurons, which transmit stimuli from the sensory receptors in the skin, skeletal muscles, joints, and the special senses, to the CNS. In parallel, SNS motor neurons, called somatic motor neurons, voluntarily convey information from the CNS to the skeletal muscles.  

Autonomic motor neurons can be subcategorized to sympathetic and parasympathetic classes, which typically induce opposing effects. The two neurons types convey information from the CNS to smooth muscle, cardiac muscle, and glands, leading to muscle contraction and inducing glandular activity. The sympathetic autonomic motor neurons support exercise or emergency responses, "fight-or-flight" responses, while the parasympathetic division regulates "rest-and-digest" activities.
The Enteric Nervous System spans the entire length of the gastrointestinal tract (GIT) and is comprised of over 100 million neurons, which include both sensory and motor components, whose function is involuntary and CNS-independent. The ENS sensory neurons monitor both chemical and mechanical modifications within the GIT, while the motor neurons control GIT smooth muscle contraction, underlying passage of food through the GIT. These neurons are also responsible for regulating gastric acid secretion and secretion of endocrine cell-derived hormones.

\item \paragraph{Describe the lobes of cortex, according to the handout.} 
There are five lobes in the cortex: frontal lobe, parietal lobe, occipital lobe, temporal lobe and limbic lobe.
\begin{itemize}
\item frontal lobe: it is located in the frontal part of the brain, before the parietal lobe and above temporal lobe. The frontal lobe is separated from the parietal lobe by central sulcus and from temporal lobe by lateral fissure. The frontal lobe contains the primary motor cortex responsible for moviments of the body parts, also the frontal lobe is responsible for thinking, planning, etc.
\item parietal lobe: it is located above the temporal lobe and between frontal and occipital lobes. The parietal lobe contains the somatic area, responsible for sensations and perception.
\item occipital lobe: is the visual processing center of the brain, contains the primary visual cortex.
\item temporal lobe: receive and interprets auditory information. It is responsible for memory and understanding/comprehension of language.
\item limbic lobe: is localized in the medial surfaceof each hemisphere (parts with the frontal, parietal and temporal lobes). Initially, its localization consider the cingulate gyrus and the parahippocampal gyrus. However, the limbic lobe can include more or less structures (it is not totally defined yet). It is a higher neural center that coordinate emotions.
\end{itemize}

\item \paragraph{Describe the structure and function of dopaminergic systems, the relation between the systems and psychology and neuropharmacology}

\item \paragraph{Describe how your project is related to some brain parts OR talk about the midbrain and its functional parts (+labelling a drawing of it).}
Midbrains is also called mesencephalon. Please, check question \ref{question:mesencephalon}

\item \paragraph{PQ - Hindbrain: components}
The hindbrain is also called rhombenceplhalon. It can be divided in two major parts: myelencephalon and metencephalon. The myencephalon forms the medulla oblongata, and the metencephalon forms the pons and cerebellum.
\begin{itemize}
\item cerebellum: the human cerebellum does not initiate motor functions but it is related to coordination, precision and accurate timing. The cerebellum receives sensory information and integrates thses input to fine tuning of the motor activity.
\end{itemize}

\item \paragraph{PQ - Brainstem: components}
The brainstem is located at the base of the brain superior to the spinal cord. In the human brain includes the midbrain (see parts in question \ref{question:mesencephalon}), pons and medulla. The brainstem provides the main motor and sensory innervation to the face and neck: of the twelve cranial nerves, ten originates at the brainstem
\begin{itemize}
\item pons: located between the medulla oblongata and the midbrain. It contains tracts that carry signals from the cerebrum to medulla and to the cerebellum.
\item medulla (or medulla oblongata): is located in the lowest part of the brainstem. It transmits signals between the spinal cord and the higher parts of the brain.
\end{itemize}
\end{enumerate}


%%%%%%%%%%%%%%%%%%%%%%%%%%%%%%%%%%%%%%%%%%%%%%%%%%%%%%%%%%%%%%%%%%%%%%%%%%%%%%%%%%%%%%%%%%%%%%%%%%%
\subsubsection{Development}
\begin{enumerate}
\item \paragraph{Name some crucial functions of Neurotrophic factors.}
Neurotrophic factors are a family of biomolecules that support the growth, survival and diferentiation of developing and mature cells.

\item \paragraph{How do different types of neurons differentiate in the neural tube?}
See question \ref{question:cns-pns}

\item \paragraph{How do different types of neurons differentiate in the periphery?}
TODO

\item \paragraph{Example for migration}
TODO

\item \paragraph{Compare in a table with advantages and disadvantages the models: Drosophila, C. Elegans, Zebrafish, Mouse. Give an example for each of how it helped understand the brain}
TODO

\item \paragraph{Compare structure and development of the cerebellum and the cortex}
TODO

\item \paragraph{What evidence did Sperry find that supports his chemoaffinity hypothesis?}

The chemoaffinity hypothesis suggests that neurons make connections with their targets based on interactions with specific molecular markers. Sperry did the follow experiment: detached a frog eye, rotated it by 180 degrees, and then putted it back. When the axons rewired, the vision was rotated by 180 degrees. So, he create a hypothesis that there is a molecular tagging the axon grow.

\item \paragraph{Have Sperry’s proposed recognition molecules been found? If yes, name one example and describe what properties of this molecule supports it’s role as a recognition molecule.}\label{question:chemoaffinity}

Yes, one example of the molecule is the Ephrin. To find this, one experiment was organized: posterior or anterior retinal explants were presented with alternating lanes, or stripes, of membrane derived from the anterior and posterior tectum. They found that posterior axons avoided membranes form the posterior tectum, while anterior axons showed no preference for either. So, they conclude this behaviour should be related with a high concentration of repulsive substance in the posterior of the tectum.
The EphrinA is found in a high concentration in posterior and low concentration in anterior of the tectum while EphA is found in a high concentration in anterior and low concentration in posterior of the retina.

\item \paragraph{What were Sperry's findings that support the chemoaffinity hypothesis. What molecules are involved in this and how do they function.}
See question \ref{question:chemoaffinity}

\item \paragraph{Development of CNS and PNS}\label{question:cns-pns}
Neural induction is the earlist step in the determination of ectodermal fates. The BMP (bone morphogenic protein) act as a signal of epidermal induction, and this give rise to the default model of neural induction. The default model of neural induction says that ectodermal cells will become neural as long as they are not exposed to antineuralizing signals. The formation of the brain starts with a process called neurulation. The neurulation is a formation of a hollow tube (neural tube) from a flat sheet of cells (neural plate). The neural tube will form the CNS, and some neual crest cells detach during the neurulation to form the PNS.

After the neural tube is formed, the developing nervous system cells rapidly increase in number and three bulges appear at the anterior end of the neural tube and become the forebrain, midbrain, and hindbrain. Cell division occurs in the ventricular zone of the neural tube (the zone next to the ventricle); when they leave the cell division cycle, cells migrate into other layers. The cells of the neocortex migrate in an inside-out pattern; the deepest layers form first so that the cells of the superficial layers must migrate through them. Migration of the cells of the neural crest is of particular interest because these cells ultimately form the PNS, and thus many have a long way to migrate. Neural crest cells transplanted to a new part of the neural crest migrate to the destination that is appropriate for cells in the new location; thus the migration routes must be encoded in the medium rather than in the cells; differential adhesion to routes through the medium is hypothesized to guide the migration of future PNS neurons. Once migration is complete, cells must aggregate correctly to form various neural structures; this is hypothesized to be mediated by specialized neural cell adhesion molecules in the cell membranes.

In the neural tube, the differentiation in diverses parts of the brain is guided by hox genes. The type and concetration of hox genes will become a specific part of the CNS. The change in this concentration or hox genes type, for instance, to have two equals concentration and types in different places will give origin for the same part in different locations.
The notochord in the floor plate give origin to the motor neuron cells. Aditional notochor will create new motor neuron cells and the removal of the notochord will generate no motor neurons.

From roof to plate the concentration of BMP decreases. From plate to roof the concentration of SHH decrease, and this gradient of concentrations will determine the cells fate.

\item \paragraph{Pathfinding (Chemoaffinity, give 2 examples)}
TODO

\item \paragraph{Compare the cell migration to form the cortex and the migration in the peripheral neural system forming…}
TODO

\item \paragraph{What’s the experiment led to the finding of neurotrophic factor? Compare trophic and tropic factor.}
Experiment: TODO

Tropic molecules guide growing axons toward a source and trophic molecules support the survival and growth of neurons and their processes once an appropriate target has been contacted.

\item \paragraph{How does cells differentiate in neural tube? How does cells differentiate in peripheral nervous system?}
TODO

\item \paragraph{Neurogenesis areas, labelling techniques, positive regulators and diseases associated with problems in neurogenesis.}
In adults, neurogenesis occurs in two areas: SVZ (subventricular zone - olphatoy bulb) and SGZ (subgranular zone - detite gyrus - hippocampus). To map the neurogenesis we can use labelling techiniques as retroviruses labeling and BrdU labeling. Both techiniques can be used to cell lineage mapping and counting. The major advantage of the BrdU labeling is its sensitivity to detecting proliferating cells, however some conditions like inflammation may lead to a different number of counting cells. The major advantage of retroviruses labeling is its infect the entire cell, allowing to picture morphological characteristics, however, the injection of the retroviruses causes brain injuries.
In adults, there are some factors that regulates the neurogenesis. Some of positive regulators (that is, increasing the neurogenesis) are running (exercises in free will) and rich environment. In the other hand, negative regulators (decreasing the neurogenesis) are stress, depression, drugs, aging etc.
The decreasing on neurogenesis cause spatial-memory and pattern-separation problems.
Some diseases associated with problems in neurogenesis are Parkinson, Alzheimer and Huntington.

\item \paragraph{5 methods (advantages + limitations) to label CNS neurons in rodent.}
TODO

\item \paragraph{PQ - What are neural stem cells?}
Neural stem cells are cells that can self­renew (maintenance and expansion). They can give rise to neurons and glia cells.

\item \paragraph{PQ - How can adult neural stem cells be studied?} We can study adult neural stem cells to observe how the cell divide and labelling them, with retroviruses labelling for instance. Another way is promote the ablation of the neurogenesis. There are some ways to ablate neurogenesis: irradiation, transgenic mice or locall cell ablation.

\item \paragraph{PQ - What influences Adult Neural Stem cells / Neurogenesis?}
In adults, the neurogenesis can be regulated for factors as voluntary exercise (running), enriched environment that increases neurogenesis or aging, stress, depression, drugs that decreases neurogenesis.

\item \paragraph{PQ - What is the functional contribution of adult neural stem cells?}
There are some controversial results about the reason of neurogenesis in adults. The main (probably) functions are learning (hippocampus and olfactory bulb), working memory and pattern separation (where two similar inputs are transformed in less similar inputs).

\item \paragraph{PQ - Does adult neurogenesis occur in humans?}
Yes. However, there are just two areas where the adult neurogenesis occurs: SVZ (sub ventricular zone - olfactory bulb) and SGZ (sub granular zone - part of dentate gyrus of the hippocampus).

\item \paragraph{PQ - What it is a hope for degenerative diseases?} Therapy with neural stem cell transplantation. Neural stem cells can generate other type of cell and maybe regenerate some degenarated cells.

\item \paragraph{PQ - How do axons move (mechanically)? Reorganization of microtubules in the growth cone how do axons navigate?}
The axons movin using the growth cone, that "looks" in the external enviroment for signals indicating the direction to grow. These signals are cues that can be attractive or repelent (and rang in short or long length). The growth cones contains receptors that recognize these cues and interpret the signal as a chemotropic response.
Some relation between receptor and cues (ligand) are:\\
DCC and UNC $\rightarrow$ Netrins $\rightarrow$ attractive for DCC and repulsive for UNC\\
Plexins $\rightarrow$ Semaphorins $\rightarrow$ repulsive\\
Robo $\rightarrow$ Slits $\rightarrow$ repulsive\\
Eph $\rightarrow$ Ephrin $\rightarrow$ repulsive\\

\item \paragraph{PQ - How to study basic principles of axonal pathfindig?}
One way to study the axon pathfinding is direct manipulation of growth axons exposing them to purified guided cues to see if they will cause the axon to turn or not. One experiment used to check the axonal pathfinding is the stripe choice experiment, where we can see axons growing to specific regions instead of others.

\item \paragraph{PQ - Define the terms: sleep homoeostasis, circadian rhythm. Explain the physiological markers of each of these processes and describe their relevance and meaning.} 
\begin{itemize}
\item sleep homeostasis: Homeostasis refers to regulatory mechanisms that maintain the constancy of the physiology of organisms. Sleep has a regulatory system enabling organisms to compensate for the loss of sleep (e.g. due to sleep deprivation) or surplus sleep (e.g by prolonging sleep in the morning or by napping). Physiological markers is the sleep intensity. The homeostatic mechanism regulates sleep intensity (that increases according to how long it is awake), while the circadian clock regulates the timing of sleep.
\item circadian rhythm: Repetitive event with a period length of about one day. the physiological markers (in humans): melatonin secretion by the pineal gland.
\end{itemize}
\end{enumerate}

%%%%%%%%%%%%%%%%%%%%%%%%%%%%%%%%%%%%%%%%%%%%%%%%%%%%%%%%%%%%%%%%%%%%%%%%%%%%%%%%%%%%%%%%%%%%%%%%%%%
\subsubsection{Ion Channel}
\begin{enumerate}
\item \paragraph{Why is the ion channel selective to an ion with particular polarity, e.g. why the positive channel allows cations going through? List different ion channels in different part of the neuron: (1) axon, (2) axon terminal and (3) postsynaptic membrane}

Why: ??? \\
Ion channels are selectively permeable, that is, allow some types of ions going through. The selectivity usually uses the size and the charge of the ions.

\begin{itemize}
\item Axon ion channels: ion channels on axon are used to propagate the action potential. In unmyelinated axons, the action potential is propagated smoothly, however, in myelinated axon, there are ion channels only at nodes of ranvier, this way the action potential "jump" between nodes, and then is propagated faster.
\item Axon terminal ion channels: are the autoreceptors. Sometimes, the presynaptic cell release the neurotransmitters and they also bind with the ion channels on the presynaptic cell. This generates a feedback to the presynaptic cell.
\item Postsynaptic membrane ion channel: are the receptors. When the neurotransmitters are released on the synaptic cleft, they bind with the receptors (ion channels or g-proteins) and allow ions to cross the membrane.
\end{itemize}

\item \paragraph{Describe the structure of a voltage-gated potassium channel. Explain the mechanisms that make the channel selective for only potassium ions.}
A voltage-gate channel is a channel sensitive to the voltage gradient across the membrane. The voltage-gate potassium channel are selective for ions with the same charge and size of K+. The pore of potassium channel is negatively charged, attracting positive potassium. The channel has a "filter" which is size-specific for potassium.

\item \paragraph{Draw a flowchart for a typical neuroproteomics experiment}
TODO

\item \paragraph{What are the principal functions of dendrites, axon and nerves endings in the transcription of signals through the nervous system? Which types of ion channels are critical for the function of each of these 3 structures? Provide specific examples.}
\begin{itemize}
\item dendrites: contains voltage-gated ion channels, dendritic spikes have grest implications in learning, memory and neuronal communication. They are often major factors of LTP.
\item axon: also contains voltage-gated ion channels, axon spikes occurs when th EPSP or IPSP (that are summed up in axon hilock) exceeds the trigger threshold and then, the action potential propagates through the axon,
\item nerve endings: ??
\end{itemize}

\item \paragraph{What’s the difference between ionotropic and metabotropic receptors?} Both ionotropic and metabotropic receptors are transmembrane. Ionotropic receptors are made from proteins combined to form an ion channel. The neurotransmitter binds the receptor and open the channel for the ions going through. Ionotropic receptor have rapid changes with short duration. Metabotropic receptors are made of a single peptide and the neurotransmitter binds the receptor that activates a g-protein. It is need a second messenger to open the ion channel, one example of second messenger is calcium. Metabotropic receptors have slower response but with long duration.

\item \paragraph{Structural selectivity feature allowing cationic channel to conduct positive but not negative ions}
TODO

\end{enumerate}

%%%%%%%%%%%%%%%%%%%%%%%%%%%%%%%%%%%%%%%%%%%%%%%%%%%%%%%%%%%%%%%%%%%%%%%%%%%%%%%%%%%%%%%%%%%%%%%%%%%
\subsubsection{Synaptic Transmission}
\begin{enumerate}
\item \paragraph{How is information transported in the nervous system? Explain features and function.}
The information is transported either along a single neuron or between neurons. Along a single neuron with an action potential and between two neurons through synapses (chemical or electrical).
In chemical transmission, the presynaptic cell release vesicles that contains neurotransmitters in the synaptic cleft. Then, these neurotransmitters bind the receptors on the postsynaptic cell. The signal generated on the postsynaptic cell can be excitatory or inhibitory. If the signal is strong enough, the postsynaptic cell fires passing the information on.

\item \paragraph{Transmission of neuronal signals: list ion channels critical for i) Axons ii) Dendrites (difference excitatory, inhibitory) and iii) Nerve Terminals}

\begin{itemize}
\item Axons: ion channels on axon are used to propagate the action potential. In unmyelinated axons, the action potential is propagated smoothly, however, in myelinated axon, there are ion channels only at nodes of ranvier, this way the action potential "jump" between nodes, and then is propagated faster.
\item dendrites: ion channel on dendrites are the receptors. When the neurotransmitters are released on the synaptic cleft, they bind with the receptors (ion channels or g-proteins) and allow ions to cross the membrane. The signal generated on the receptor cell can be excitatory or inhibitory. If the signal is strong enough, the postsynaptic cell fires passing the information on.
\item Nerve terminal ion channels: are the autoreceptors. Sometimes, the presynaptic cell release the neurotransmitters and they also bind with the ion channels on the presynaptic cell. This generates a feedback to the presynaptic cell.
\end{itemize}

\item \paragraph{Function of AP (action potential), resting potential, synaptic potential, channels for all these potentials}

\begin{itemize}
\item action potential: electrical signal produced on axon to boost the information flow in the neuron.	Main function is make the information going througth the axon, once the neuron is not naturally a good eletrical conductor.
\item resting potential: is the point of equilibrium wrt the ions to which the membrane is permeable. If the membrane is permeable just to one ion, then the resting potential will be equal to equilibrium potential for this ion. If the membrane is permeable to more than one ion, the resting potential is a value between all the individual equilibrium potential. The main function of the resting potential is be a threshold who indicates when a neuron fire or not.
\item synaptic potential: electrical signal associated with communication between neurons. The main function of the synaptic potential is allow transmission of information from one neuron to another.
\end{itemize}

\item \paragraph{What are the two main modes of electrophysiological communication between neurons? Describe structure.}\label{question:synaptic-communication}
The two main modes of communication between neurons are chemical and electrical.
\begin{itemize}
\item chemical: this kind of communication occurs throught synaptic clefts. This communication is a sequential communication: when the information arrives the axon terminals, the vesicles are merged with the membrane and release neurotransmitters. These neurotransmitter will bind the receptors on the postsynaptic cell to flow the information. The action potential is generated because of the ions movement. The membrane usually has a resting potential around -70mV (is said polarized). When the membrane is polarized, the ion channels are closed and there is no ion movement. When the receptor bins, the channels open and the ions move, depolarizing the cell. If the depolarization reachs a threshold, the action potential is initiated by complete depolarization. After the spike, the ion channels close and the membrane initiates the repolarization. The output of a chemical communication is the release of neurotransmitters.
\item electrical: electrical communication occurs through gap junctions. In the gap junctions, neurons fire together. Electrical communication is faster than chemical communication (usually find on the need of really faster responses, as on defensive reflexes), however, electrical communication lacks gain, that is, the action potential of the post synaptic cell is equal or less the action potential of the presynaptic cell. One difference between the chemical and electrical communication is the electrical does not need a receptor to recognize chemical messages. The output of a electrical communication is the electrical signal itself.
\end{itemize}

A good material to better understand the communication between neurons is this \href{http://www.mind.ilstu.edu/curriculum/neurons_intro/neurons_intro.php}{site}.

\item \paragraph{Chemical + electrical synapses}
see question \ref{question:synaptic-communication}

\end{enumerate}

%%%%%%%%%%%%%%%%%%%%%%%%%%%%%%%%%%%%%%%%%%%%%%%%%%%%%%%%%%%%%%%%%%%%%%%%%%%%%%%%%%%%%%%%%%%%%%%%%%%
\subsubsection{Auditory System}
\begin{enumerate}
\item \paragraph{Describe how sound is encoded neurally (from entering the ear to being perceived as sound in brain - complete pathway)}\label{question:auditory-pathway}

The sound generates variation in air pressure that enter the external and the middle ear. Then, the middle ear amplifies the vibration to send to the inner ear because the innear ear is filled with fluid. After that, the hair cells in the cochlea transduce the sound waves in neural signals. In the cochlea the complex souns are also decomposed into simple elements. After that, the signal reaches the auditory never fibers that branches and enters the CNS through the cochlear nucleus. After the cochlear nucleus, the signal enters the superior olivary complex. In the supperior olivary complex (medial superior complexs, lateral superior complex and medial nucleus trapezoid body) the signal from both ears sit together, this way, in the MSO it is computed the interaural time difference (the MSO cells work as a coincident detector, responding when the signal from both ears arrive together) and in the LSO it is computed the interaural level difference (intensitie). The medial nucleus trapezoid body send inhibitory signals to the LSO, this way, the input from LSO are excitatory ipsilateral signals and inhibitory contralateral signals. After that, the signals enters the inferior colliculus, where the input is the ITD and ILD together with the tonotopic map, tuning the spatial location, then goes to the medial geniculate body (thalamus) where the tonotopic map is preserved in the ventral nucleus of MGB and finally, arrive in the auditory cortex. The auditory cortex is divided in core(primary), belt (secondary) and parabelt (higher level). In the primary, we can find a tonotopic map This pathway is show in Figure \ref{fig:auditory-pathway}.

When the hair cells boundle in direction of the tallest cillium transduction open K+ channels and depolarization occurs, so, Ca2+ voltage-gate channels open and the transmitters are released onto the auditory nerve.

The primary auditory cortex is located in the superior temporal gyrus and has a precise tonotopic map (representation of the sound frequency). The dorsal stream is responsible for spatial sound processing (where) and the ventral stream for sound identification (what).

The descending pathway for the auditory system it is used to minimize damage to loud sounds, setting cochlear gain and as a negative feedback, inhibiting some signals, reducing noise.

\begin{figure}[H]
	\centering
  	\includegraphics[width=0.8\linewidth]{imgs/auditory-pathway.jpg}
	\caption{Auditory pathway}
  	\label{fig:auditory-pathway}
\end{figure}

\item \paragraph{Auditory pathway}
see question \ref{question:auditory-pathway}

\item \paragraph{Describe differences between "conductive hearing loss" and "sensorineural hearing loss". Describe the classical test which is often used to determine between both forms of hearing loss. Describe biological causes and current treatments aids for such hearing impairments.}
\begin{itemize}
\item Conductive hearing loss: involve damage to the external or middle ear, can be caused, for example, by occlusion of the ear canal by wax or external objects, rupture of tympanic membrane. The effect of this loss is the sound seems faint.
\item Sensorineural hearing loss: involve damage to the inner ear, mostly tipically the hair cells or the 8th nerve itself. Usually, this damage is caused by a congenital factor or environmental insults that leads to cell death.
\end{itemize}
A classical test to distinguish between these two forms of hearing loss is the Weber test. This test can be conducted if we previously know which ear has a injury. With a vibrating fork localized in the middle of the both ears, the pacient is asked to say which ear the sound is louder. In a patient without loss, the sound will be identified as the same level in both sides. If the patient report the sound louder in the damaged ear, it is the conductive hear loss case. However if the patient reports the sound is better in the good ear, than it is the sensorineural hearing loss case. This happens because in the conductive loss, the sound do not dissipate freely and then a great amount of sound is transmitted to the cochlea, while in the sensorineural hearing loss the vibration of the sound is not transduced into a neural signal.

\item \paragraph{Bird auditory neuron behaviors, setup of the experiment with the bird and why is it important that the bird does not hear any external sounds.}

\end{enumerate}

%%%%%%%%%%%%%%%%%%%%%%%%%%%%%%%%%%%%%%%%%%%%%%%%%%%%%%%%%%%%%%%%%%%%%%%%%%%%%%%%%%%%%%%%%%%%%%%%%%%
\subsubsection{Sensory-Motor System}
\begin{enumerate}
\item \paragraph{Describe how DRG (dorsal root ganglion) sensory neurons development in comparison to motor neurons. How are cell boundaries formed in general and among the specific motor/sensory nerves}

\item \paragraph{Sensory inputs contribute to the production of locomotor patterns in three important ways. Identify these functions and illustrate them by means of clear examples.}

\item \paragraph{PQ - Motor pathway}

\item \paragraph{PQ - Sensory pathway}

\item \paragraph{PQ - Basal ganglia loops}

\item \paragraph{PQ - Direct and indirect pathways of basal ganglia}
\end{enumerate}

%%%%%%%%%%%%%%%%%%%%%%%%%%%%%%%%%%%%%%%%%%%%%%%%%%%%%%%%%%%%%%%%%%%%%%%%%%%%%%%%%%%%%%%%%%%%%%%%%%%
\subsubsection{Visual System}\label{sec:visual-system}
\begin{enumerate}
\item \paragraph{Describe the two columns in V1, and give out the definitions of them. What is the relation between this two columns and also between them and other parts of visual system. give out one experiment for testing of each column.}
In V1 we find two kind of columns: the orientation column and the ocular dominance column.
\begin{itemize}
\item orientation column: areas that selectively respond to specific orientation in the visual field. Experiment: present stimulli for an animal using different orientations, using a video camera record changes in light absorption. For each orientation stimullu use a color to combine all the recorded images. The combination will present a pattern as a pinwheel.
\item ocular dominance column: (also called ocular preference), some areas that are selective for one eye only. Experiment: close only the right eye from the animal and present stimulli to the left eye. Using a sensitive video camera record changes in light absorption that occur as the animal views the stimuli. After, close only the left eye of the animal and repeat the process. The difference between the two recorded images will appear with a stripped-shape, showing the ocular dominance.
\end{itemize}

The relation between the two columns is determined by the organization of the visual cortex. If you move the electrode tangentially through the cortex, you will find the ocular dominance column. However, if you move the electrode tangentially in the orthogonal direction you will find the orientation dominance.

Relation with other areas: ???

\item \paragraph{Describe three functional properties of neurons in area V1 that are absent in the Lateral Geniculate Nucleus. For each property describe in detail an experiment that illustrates it, including the type of stimulus and the observed neuronal responses. Finally, choose one these three properties and explain as precisely as possible how it can emerge at the cortical level.}

\begin{itemize}
\item binocular depth (stereopsis): in the V1 area, neurons have information about both eyes together, this way, they can be sensitive to the depth wrt the fixation point. Experiment: fix your finger in front of your eyes (with a distance about 10 centimeters. In an alternated way, open and close each eye (right eye closed and left open and vice-versa). You will see your finger in different positions. However, with both eyes open you see just one image.
\item orientation: Some V1 neurons are orientation selective - they respond strongly to lines, bars, or edges of a particular orientation (e.g., vertical). Experiment: an animal is fitted to focus the eyes on a screen, where oriented images will be projected. An extracellular electrode records the neuronal responses. We will see that some neurons will respond strongly for lines/bar/edges in specific orientation and will respond weakly (or even not respond at all) to other orientations.
\item direction (motion): Some V1 neurons are direction selective - they respond strongly to oriented lines/bars/edges moving in a preferred direction (e.g., vertical lines moving to the right). Experiment: the same experiment can be applied for direction. However, the oriented line/bar/edge will move for an specific direction this time. We will see that some neurons will respond strongly for lines/bar/edges moving in a specific direction and will respond weakly (or even not respond at all) to other directions.
\end{itemize} 

How binocularity emerge at the cortical level: each eye see the world from a slightly different point of view, so, objects that are before or after the fixation point are projected to noncorresponding points on the retinas. There are some neurons in V1 area that are maximally activated when the stimulus fall on noncorresponding points on the retina. They are the near cells and the far cells. The near cells discharge to disparities in front of the fixation point and the far cells discharge to disparities beyond the fixation point. The combination of these neurons contribute to the depth sensation.

\item \paragraph{In verterbrates the vision system has some special wiring pattern. What's special about it (as in, how is it different to olfaction)? Explain biological/physiological means in the development of vision.}

In visual system, there is a topographic map on the cortex (the spatial arrangement in the retina is preserved in the primary visual cortex). This way, the image observed for the retina is mapped into the cortex in the same way. But the left hemisphere maps the image from the right retina and vice versa. In the visual cortex we have a map of location of the sensory stimulus as well as a quality of the stimulus.
For the olfaction, a spatial map is not necessary, so, the map on olfactory system is regarding the quality of an odorant stimulus

\item \paragraph{Imagine year 2020. Human genomics has advanced to the point where you
not only can choose the gender and hair color of your child, but also apply specific
changes to the visual system. Name 6 changes to the human visual system you would apply to your kid. Explain why you chose them and what physiological implications they would have.}\label{question:year2020}

\begin{itemize}
\item add new subtype of cones to see other wavelength (much more colors and even infrared)
\item change the way how the ganglion axons exits retina, to avoid a blind spot.
\item add other subtypes of rod (to add color), this way, with low light the color will be still present.
\item 
\item
\item
\end{itemize}
%https://www.youtube.com/watch?v=T9HYPlE8xzc

\item \paragraph{Insect eye} 

\item \paragraph{Vision:}
\subparagraph{A. cites six roles for vision in insects}
\subparagraph{B. 1. what structural/functional differences between insect and human eyes 2. in what experiments are insects eyes worse or better than human eyes? 3. five share important features of insect and human eyes}
\subparagraph{C. 1. what is flow field? 2. draw the flow field perceived by a fly flying straight in a long corridor 3. what is ... field? ( i forgot the name oops) 4. draw pure rotation force field and its matched field filter} TODO

\item \paragraph{PQ - Visual System pathway}

The image (visual fiedl) is inverted and reversed in the retinal image. The retina is a neural portion of the eye, contains five type of neurons:
\begin{itemize}
\item photoreceptors: do not exhibit action potentials, light activation causes a graded change in membrane potential and a corresponding change in the rate of transmitter release. Action potential are not necessary to transmit information over so short distance. They do phototransduction.
\subitem rhods: scotopic vision, low light
\subitem cones: photopic vision, more light. Three types: Short (Blue), Medium (G) and Long (R) wavelets.
\item bipolar cells
\item ganglion cells
\item horizontal cells
\item amacrine cells
\end{itemize} 

Phototransduction: the light in a photoreceptor causes hyperpolarization. Transmitter release from synaptic of photoreceptor is dependent on voltage-sensitive Ca2+ channels. When is dark, the Ca2+ channels open and the Ca2+ enters the membrane, increasing the number of transmitters released. When there is light, the Ca2+ channels open but the Ca2+ exits the membrane, decreasing the number of transmitters released.

The axons of the retina ganglion cells form the optic nerve and go straight to the optic chiasm. In the optic chiasm, the nasal axons cross the midline and together with the temporal axons form the optical tract (with fibers of both eyes). Here, the fibers take different branchs:
\begin{itemize}
\item to the lateral geniculate nucleus (thalamus), that is compost of six layers\footnote{2 layers are of magnocelular - large neurons, and 4 are of parvocelular - small neurons. The layers are in the following order: contralateral-nasal, ipsilateral-temporal, ipsilateral-temporal, contralateral-nasal, ipsilateral-temporal, contralateral-nasal}.
\item to superior colliculus: head and eye moviments to visual targets
\item to pretectum: puppilary light reflex
\end{itemize}

From the Lateral Geniculate Nucleus, the signal flows to the optic radiation (internal capsule) and then to the Primary Visual Cortex. Figure \ref{fig:visual-pathway} represent this flow.

\begin{figure}[H]
	\centering
  	\includegraphics[width=0.8\linewidth]{imgs/visual-pathway.jpg}
	\caption{Visual pathway}
  	\label{fig:visual-pathway}
\end{figure}


\end{enumerate}

%%%%%%%%%%%%%%%%%%%%%%%%%%%%%%%%%%%%%%%%%%%%%%%%%%%%%%%%%%%%%%%%%%%%%%%%%%%%%%%%%%%%%%%%%%%%%%%%%%%
\subsubsection{Vestibular System}

\begin{enumerate}
\item \paragraph{PQ - Vestibular System pathway}
Four main functions of vestibular system:
\begin{itemize}
\item perception of self motion
\item head position
\item spatial orientation (relative to gravity)
\item motor function: stabilization of eyes, head and posture.
\end{itemize}

Five organs:
\begin{itemize}
\item Semicircular organs: related with rotation and acceleration of the head
\subitem horizontal canal
\subitem anterior canal
\subitem posterior canal
\item Otolith organs: related with linear acceleration of the head/static head positions 
\subitem urticle: hair cells organized in different directions, but towards the striola 
\subitem saccule: hair cells organized in different directions but outwards the striola
\end{itemize}

The pair of semicircular organs are organized as: the two horizontal canals and the anterior ipsilateral canal with the posterior contralateral canal and the anterior contralateral canal with the posterior ipsilateral canal.

The hair cells are located at urticle, saccule and ampulla (in the ampulla, the hair cells are all polarized to the same direction).
When the head (or body) moviments, the hair cells in the vestibular organs can be activated (depolarized), and then they send a signal to the vestibular nerve. The superior vestibular nerve receives input from horizontal canal, anterior canal and urticle. The inferior vestibular nerve receives input from posterior canal and saccule. So, the signal is sent to vestibular nuclei and then to the thalamus (vestibular posterior nuclei complex), where they can projected to primary somatosensory cortex or to an area between somatic sensory cortex and motor cortex. Figure \ref{fig:vestibular-pathway} shows a simplified flow.

There are three reflex of the vestibular system:
\begin{itemize}
\item vestibulo-ocular reflex: maintain equilibrium and eye fixed during head moviments. 
\subitem head rotation: the moviment of the eyes are in two phases: slow (until the limit of orbital range) and fast: jump moviment for a new position.
\subitem head translation (linear moviments): near objects move fast, far objects move slow
\subitem head moviment in the vertical plane: orientation relative to the gravity
\item vestibulo-cervical reflex: maintain posture
\item vestibulo-spinal reflex: maintain muscle tone
\end{itemize}

balance is the product of vestibular, vision and proprioception inputs.

\begin{figure}[H]
	\centering
  	\includegraphics[width=0.8\linewidth]{imgs/vestibular-pathway.jpg}
	\caption{Vestibular pathway}
  	\label{fig:vestibular-pathway}
\end{figure}

\end{enumerate}

%%%%%%%%%%%%%%%%%%%%%%%%%%%%%%%%%%%%%%%%%%%%%%%%%%%%%%%%%%%%%%%%%%%%%%%%%%%%%%%%%%%%%%%%%%%%%%%%%%%
\subsubsection{Learning and Memory}

\begin{enumerate}
\item \paragraph{Everything about declarative and nondeclarative memory:}
A lot of behaviors come from learning. Learning is acquisition of knowledge or skills through experience. The long term memory is divided in two parts: declarative and non declarative memory.
\subparagraph{a. short definitions}
\begin{itemize}
\item declarative memory (or explicit memory): is the memory based in facts and events. Involves conscious recollection, like remember the time of an appointment.
\item nondeclarative memory (or implicit memory): is used unconsciously and can affect thoughts and behaviors. It is divided in four major parts: 
\begin{itemize}
\item priming memory: is the memory of recent things (we tend to believe in recent acquisition more truthly).
\item procedural memory: is our skills and habits, we access everyday to make our way to work or to lace a shoe, for instance.
\item associative memory: is subdivided in endronal and skeletal.
\subitem endronal associative memory: related with remember the face of a friend when we hear his/her voice.
\subitem skeletal associative memory: related with an action, as when you salivate when you see your favorite food.
\item The nonassociative memory: is subdivided in habituation and sensitization
\subitem habituation: decrease of response to a repetitive stimulus, when you do not care anymore with a repetitive noise
\subitem sensitization: increase of response to various stimulus after a noxious one, when you do not have more courage to jump after break a leg once.
\end{itemize} 
\end{itemize}
\subparagraph{b. anatomical location}
\begin{itemize}
\item declarative memory: medial temporal lobe
\item nondeclarative memory
\begin{itemize}
\item priming: neocortex
\item procedural: striatum
\item endronal associative: amygdala
\item skeletal associative: cerebellum
\item nonassociative (habituation and sensitization): reflex pathways
\end{itemize}
\end{itemize}
\subparagraph{c. animal models} ??

\item \paragraph{Please give a definition of "classical conditioning" and explain in the terms US, CS and CR. Give at least two examples for rodent procedures of classical conditioning and specify for each of them what the US, CS and CR are.}
Classical conditionin is a learning procedure in which a good or bad stimullus is paired with a neutral stimulus in order to produce a new behavior. The US (unconditioned stimulus) is the good or bad stimulus, CS (conditioning stimulus) is the neutral stimulus and CR (conditioned response) is the result of the learning.
\begin{itemize}
\item example one: the mouse has no reaction when a tone is played (play tone = unconditioned stimulus). When the rat receives a shock, it freezes (shock = conditioned stimulus). During training phase, we will play a tone and soon after we will give a shock in the rat. After the training phase, when we play the tone, the rat will freeze, even if no shock is given, this is the conditioned response.
\item example two: 
\end{itemize} 

\end{enumerate}

%%%%%%%%%%%%%%%%%%%%%%%%%%%%%%%%%%%%%%%%%%%%%%%%%%%%%%%%%%%%%%%%%%%%%%%%%%%%%%%%%%%%%%%%%%%%%%%%%%%
\subsubsection{Circuits underlying emotions}

\begin{enumerate}
\item \paragraph{Describe the potentiated startle reflex and underlying anatomical structures - fear} ?
\end{enumerate}

%%%%%%%%%%%%%%%%%%%%%%%%%%%%%%%%%%%%%%%%%%%%%%%%%%%%%%%%%%%%%%%%%%%%%%%%%%%%%%%%%%%%%%%%%%%%%%%%%%%
\subsubsection{Neuromorphic engineering}
\begin{enumerate}
\item \paragraph{Differences in organizing principles between electronic and neural computation}
\item \paragraph{Illustrate neural computation principles by specific examples (e.g. retina), explain functional utility}
\item \paragraph{Neuromorphic engineering}
\item \paragraph{Considering organizing principles used in biological retina explain (...)}
\item \paragraph{Fill in the blank and multiple choice questions from Tobi's lecture: Who invented the term Neuro Engeneering? What is CMOS? Power consumption of brain. Synchronous logic is ubiquitous slide know physiologists friend photodiodes - how they are similar to retina CARVER MEAD}
\end{enumerate}

%%%%%%%%%%%%%%%%%%%%%%%%%%%%%%%%%%%%%%%%%%%%%%%%%%%%%%%%%%%%%%%%%%%%%%%%%%%%%%%%%%%%%%%%%%%%%%%%%%%
\subsubsection{Neural networks}
\begin{enumerate}
\item \paragraph{Which dynamic process occur in a single neuron and in the local neural circuit during signal flow through a neuronal network? Name critical structural and functional aspects and discuss how they can be measured experimentally.}

\item \paragraph{Explain the temporal and spatial network definition. Give an example for each network definition and describe how you can detect these networks in the brain.}

\item \paragraph{Model organisms for understanding human brain development and function. Give three general advantages of these models. Compare in a table with advantages and disadvantages the models: Drosophila, C. Elegans, Zebrafish, Mouse. Give an example for each of how it helped understand the brain}

\end{enumerate}

%%%%%%%%%%%%%%%%%%%%%%%%%%%%%%%%%%%%%%%%%%%%%%%%%%%%%%%%%%%%%%%%%%%%%%%%%%%%%%%%%%%%%%%%%%%%%%%%%%%
\subsubsection{Neural computation}

\begin{enumerate}

\item \paragraph{Describe the perceptron learning rule and example where it can be applied} 
Perceptron is a suppervised learning, it can be applied to predict an output given the patterning input and the desired output. In the training, a output is generated for the given input pattern, then, the connection's weight are changed accordly with the product between:
\begin{itemize}
\item the difference between the desired output and the predicted output.
\item the input pattern 
\item the learning rate
\end{itemize}
The learning rule of the perceptron is given by $\Delta w = \eta *(y\prime - y) * x$, where $y\prime$ is the desired output, $y$ is the predicted output, $\eta$ is the learning rate, $x$ is the input pattern and $\Delta w$ is the weight variation.

One example were we can apply the perceptron is the prediction of handwriting digit recognition. Given examples of handwriting digits and the desired outpput (the correct number), a perceptron network can predict the desired number for a new pattern.
\end{enumerate}

%%%%%%%%%%%%%%%%%%%%%%%%%%%%%%%%%%%%%%%%%%%%%%%%%%%%%%%%%%%%%%%%%%%%%%%%%%%%%%%%%%%%%%%%%%%%%%%%%%%
\newpage

\section{References}
The pictures used in this summary are from the class slide sets or internet, and belong to their respective owners. In the context of the summary they are used for educational purposes only.

\subsection{Molecular and Cellular Neuroscience}
\begin{itemize}
\item Building a central nervous system (Neuhauss and Jessberger)
\subitem Fundamental Neuroscience Part III, Chapters 13-16 (21)
\item Excitability (Müller)
\subitem Principles of Neural Science Part II, Chapter 5-7
\subitem Fundamental Neuroscience Part II, Chapter 5
\item Glia and more (Weber)
\subitem Fundamental Neuroscience Part II, Chapters 3, 12
\subitem Principles of Neural Science Part II, Chapter 4
\item Synapses (1/2) (Földy and Karayannis)
\subitem Fundamental Neuroscience Part II, Chapters 6-8 (Part 1) and 9-11 (Part 2)
\end{itemize}

\end{document}
