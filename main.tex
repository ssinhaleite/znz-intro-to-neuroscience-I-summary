\documentclass[11pt,article,oneside,a4paper]{memoir}

%% Packages
%% ========

%% many common packages
\input{commonpackages}

%% Some more packages that you may want to use.  Have a look at the
%% file, and consult the package docs for each.
\input{extrapackages}

%% Our layout configuration.
\input{layoutsetup}

%% Theorem environments.  You will have to adapt this for a German
%% thesis.
\input{theoremsetup}

%% Helpful macros.
\input{macrosetup}

%%page layout settings and listing templates etc.
\input{settings}

\title{\textbf{ZNZ HS16 Introduction to Neuroscience I} \\
       Fall 2016\\\normalsize version 1.0}

\author{
	Vanessa Leite
	\vspace{2em}
	\\Repository page: \url{https://github.com/ssinhaleite/znz-intro-to-neuroscience-I-summary}\\
	Contact \href{mailto:vrcleite@gmail.com}{vrcleite@gmail.com} if you have any questions.}
	\thesistype{The Summary of the lectures in 2016}
	\department{ZNZ - Institute of Neuroinformatics, ETH}
	\date{\today}

\begin{document}
\frontmatter


%% DO NOT CHANGE.
\begin{titlingpage}
  \calccentering{\unitlength}
  \begin{adjustwidth*}{\unitlength-24pt}{-\unitlength-24pt}
    \maketitle
  \end{adjustwidth*}
\end{titlingpage}

\mainmatter

%% This change is needed if the article option for the memoir document class
%% is used, in order to count sections (article) as if they were chapters (memoir)
\counterwithout{section}{chapter}

%% Our content

\newpage
\clearpage
\pagenumbering{roman}
\setcounter{tocdepth}{3}
\setcounter{secnumdepth}{2}
\tableofcontents

\clearpage
\pagenumbering{arabic}

\newpage

\section{Human \& Comparative Neuroanatomy}
\subsection{Introduction}
\subsection{Human Neuroanatomy}
\subsection{Comparative Neuroanatomy}


\section{Molecular \& Cellular Neuroscience}
\subsection{Building a central nervous system}
\subsection{Excitability}
\subsection{Glia and more}
\subsection{Synapses}

\section{Systems Neuroscience}
\subsection{Somatosensory and Motor Systems}
\subsection{Visual System}
\subsection{Auditory \& Vestibular System}
\subsection{Circuits underlying Emotion}
\subsection{Learning in artificial and biological neural networks}

\newpage

\subsection{References}
The pictures used in this summary are from the following books and slide sets
and belong to their respective owners. In the context of the summary they are
used for educational purposes only.
\begin{itemize}
\item
\item
\end{itemize}
\end{document}
